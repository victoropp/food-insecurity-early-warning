% Chapter 5: Discussion and Limitations (15-18 pages)

\section{Summary of Key Findings}

This dissertation addresses a fundamental methodological challenge in news-based early warning systems: \textit{how do we know when news features provide genuine predictive value versus merely capturing the autocorrelation already present in crisis data?} Through rigorous baseline comparisons, two-stage residual modelling, ablation studies across 8 model variants, and three-method interpretability analysis, we provide empirical answers to five core research questions. This section synthesises our key findings before exploring their theoretical and practical implications.

\subsection{RQ1 Answered: The Autocorrelation Trap Quantified}

\textbf{Research Question}: To what extent can spatio-temporal autoregressive baselines replicate the performance of news-based forecasting models, and what does this reveal about the value of text features in crisis prediction?

\textbf{The Finding}: The AR baseline achieves AUC=0.907, Precision=0.732, Recall=0.732, and F1=0.732 at h=8 (32 weeks, 8 months ahead) using \textit{only two autoregressive features}---Lt (temporal autoregressive features capturing past IPC values) and Ls (spatial autoregressive features capturing neighboring districts' IPC values within 300km inverse-distance weighting)---with \textbf{zero news features, zero external covariates, and zero text data}.

This performance represents 93.8\% of the published news-based model from Balashankar et al. (2023, \textit{Science Advances}), which used 11.2M news articles to achieve PR-AUC=0.8158 across 21 countries. Our AR baseline achieves PR-AUC=0.7652 using \textit{zero news features}. To put this in further perspective: Stage 2 news models trained on AR-difficult cases (6,553 observations where IPC\textsubscript{t-1} $\leq$ 2 AND AR predicted non-crisis) achieve AUC=0.697-0.727 on this deliberately filtered, high-difficulty subset. This demonstrates that persistence patterns capture the majority of predictable crisis signal, with news features providing concentrated marginal value for specific shock-driven contexts rather than universal improvement.

\textbf{Why This Matters}: Most existing literature on news-based crisis prediction reports AUC scores between 0.75-0.90 without AR baseline comparisons \citep{balashankar2023toward, qazi2020geo, mueller2021quantifying}. Our results demonstrate that such performance can be achieved through temporal and spatial autocorrelation alone, without learning any genuine dynamic signals from text. The field's claims about ``the value of news for early warning'' require fundamental reassessment when proper baselines reveal that spatio-temporal persistence dominates prediction.

\textbf{The Implication for Literature}: If 93.8\% of Balashankar et al.'s published news model performance comes from simple persistence (``tomorrow will resemble today, and districts will resemble their neighbours''), then news features contribute at most 6.2\% marginal value beyond autocorrelation. This is the \textbf{autocorrelation trap}: models trained on temporally and spatially autocorrelated outcomes (like IPC phases) inherit persistence without necessarily learning new signals. Any claimed ``news model performance'' must be evaluated \textit{relative to AR baselines}, not absolute AUC scores.

\textbf{Where Dynamic Signals Provide Complementary Value---The Critical 1,427 Cases}: While achieving 0.907 AUC, the AR baseline identifies 1,427 crises out of 5,322 total (26.8\%) as requiring complementary signals beyond temporal and spatial persistence (analysed at optimal balanced P=R threshold 0.629). These cases exhibit systematic patterns, concentrating in:

    \begin{itemize}
    \item \textbf{Conflict-affected regions}: Sudan (230 cases, 16.1\%), Nigeria (168 cases, 11.8\%), where rapid escalations introduce dynamics beyond temporal patterns.
    \item \textbf{Economic crisis zones}: Zimbabwe (265 cases, 18.6\%), where structural transitions produce sudden IPC changes distinct from gradual trends.
    \item \textbf{Pastoral zones}: Kenya (242 cases, 17.0\%), Ethiopia (149 cases, 10.4\%), where high mobility introduces unique dynamics and climatic shocks produce rapid-onset transitions requiring additional signal sources.
    \end{itemize}

These 1,427 cases represent the \textbf{high-frequency component} of crisis dynamics---rapid-onset shocks, regime transitions, and structural breaks where persistence-based forecasting benefits from complementary news-based signals. They present greater forecasting complexity (requiring dynamic signals beyond Lt/Ls) while offering \textbf{substantial humanitarian value} (where 8-month early warnings enable preemptive interventions that reactive monitoring cannot provide).

\subsection{RQ2 Answered: When News Matters---Feature Engineering Insights}

\textbf{Research Question}: What is the role of different kinds of news features (conflict, displacement, economic, weather) and dynamic transformations (ratio vs z-score) in predicting food insecurity beyond autoregressive baselines?

\textbf{Ratio Features Achieve Higher Standalone Performance Than Z-Scores}: Ablation studies reveal that ratio features (measuring relative emphasis: conflict articles / total articles) achieve AUC=0.727$\pm$0.165 when combined with location metadata, compared to z-score features (measuring anomalies relative to historical mean) at AUC=0.699$\pm$0.165. This 0.028 AUC difference (p<0.05 via paired t-test across 5 folds) suggests that \textit{compositional shifts in news narratives} provide more robust signals than \textit{volume anomalies} for standalone model performance on AR-filtered cases.

Why ratios win: Ratios capture persistent emphasis shifts (e.g., conflict coverage rising from 10\% to 40\% of all news, sustained over months) while z-scores capture spikes that may be transient or reflect seasonal patterns. For 8-month horizon prediction, sustained narrative shifts matter more than short-lived volume spikes.

\textbf{News Themes Ranked by Cross-Method Consensus} (see Figure \ref{fig:ch4_news_themes} in Chapter 4 for comprehensive visualisation across XGBoost, Mixed Effects, and SHAP):

    \begin{enumerate}
    \item \textbf{Measurement-dependent theme rankings}: Weather emerges as strongest for ratio/mixed-effects models (+26.7 coefficient) capturing sustained compositional shifts for 8-month forecasts, while conflict dominates SHAP z-score analysis (\#1 at 0.911) for rapid anomaly detection. This measurement paradox reveals different predictive mechanisms: ratios/mixed-effects favour weather/displacement/food security for slow-onset agricultural crises; z-scores/SHAP favour conflict/humanitarian/governance for rapid shock-driven crises. Three mechanisms explain weather's dominance in compositional models: (1) Direct causal pathway$\times$weather directly causes agricultural disruption and food scarcity, (2) New information$\times$weather shocks are temporally variable unlike baseline conflict which is autocorrelated with location metadata (country\_baseline\_conflict already captures 9.3\% separately), (3) High signal-to-noise$\times$weather reporting is factual and descriptive (rainfall measurements) versus noisy conflict coverage.

    \item \textbf{Displacement news} (+21.2 coefficient, 11.2\% SHAP): Refugee flows and internal displacement serve as leading humanitarian indicators. Population movements signal deteriorating conditions before IPC assessments capture outcomes, providing genuine early warning value. Displacement operates as both crisis driver (disrupts livelihoods) and crisis indicator (people flee).

    \item \textbf{Food security news} (+20.3 coefficient, 11.8\% SHAP): Direct reporting on famine warnings, malnutrition, and humanitarian assessments. High correlation with IPC by design (journalists cover crisis declarations), but provides signal between 4-month IPC assessment periods when conditions evolve.

    \item \textbf{Conflict news} (+19.6 coefficient, 10.9\% SHAP): Violence, insurgency, civil war. Moderate importance despite theoretical salience because conflict is highly autocorrelated with baseline risk$\times$country\_baseline\_conflict metadata already captures structural conflict propensity, leaving limited marginal contribution for dynamic conflict news. Conflict matters most where it represents NEW escalations (Sudan Apr 2023, DRC M23 resurgence), not baseline violence.

    \item \textbf{Health news} (+18.6 coefficient, 9.1\% SHAP): Disease outbreaks, cholera epidemics, malnutrition rates. Health crises compound food insecurity via household income shocks but operate through indirect mechanisms with longer causal lags.

    \item \textbf{Economic/Governance/Humanitarian news} (+14-17 coefficients): Moderate predictive value, context-specific importance. Economic news matters for Zimbabwe hyperinflation crises, governance news for policy/institutional shifts, humanitarian news reactive rather than predictive.

    \item \textbf{Location features dominate tree splits, not marginal predictions}: Three country-level metadata features (data density 13.3\%, baseline conflict 9.3\%, baseline food security 6.7\%) collectively account for 40.4\% of tree-based split frequency but only 2.6\% of SHAP attribution (15.5$\times$ overstatement). This measurement paradox reveals tree-based importance captures stratification utility (geographic segmentation), while SHAP captures predictive contribution. Z-score features account for 74.7\% of SHAP attribution$\times$dynamic news anomalies drive shock-driven crisis predictions, not static geographic context.
    \end{enumerate}

\textbf{The Paradox of News Value}: Despite conflict, displacement, and food security news ranking highest among Stage 2 features, XGBoost models trained on AR-filtered cases achieve AUC=0.696-0.727 on the deliberately challenging subset (6,553 observations where IPC\textsubscript{t-1} $\leq$ 2 AND AR predicted non-crisis). This apparent paradox is resolved by recognising that news features provide value \textit{selectively}---for the 1,427 AR failures where temporal/spatial patterns break down, not universally across all 20,722 observations. Stage 2 models target shock-driven dynamics in contexts where persistence models fail, explaining why their AUC on this filtered, high-difficulty subset differs from the AR baseline's AUC on the full dataset (0.907), which includes the 73.2\% of easier persistence-driven cases.

\subsection{RQ3 Answered: The Role of Hidden Variables---HMM and DMD}

\textbf{Research Question}: What is the contribution of latent regime detection (HMM) and temporal pattern extraction (DMD) in identifying crises that autoregressive models miss?

\textbf{HMM and DMD Capture Hidden Crisis Dynamics}: Ablation studies reveal complementary contributions from latent variable models:
    \begin{itemize}
    \item Adding HMM features to basic ratio+z-score+location: AUC increases from 0.696 to 0.703 (+0.007, p=0.08), demonstrating that regime detection adds signal beyond raw article counts.
    \item Adding DMD features to basic ratio+z-score+location: AUC increases from 0.696 to 0.698 (+0.002), extracting temporal evolution patterns from news narratives.
    \item Combining HMM+DMD with all features: AUC=0.697, with feature importance analysis revealing these methods contribute through interpretability and marginal predictions (z-scores account for 74.7\% SHAP attribution) rather than standalone AUC improvement.
    \end{itemize}

\textbf{Critically, HMM and DMD provide unique signal that simpler features cannot capture}: they detect \textit{qualitative narrative shifts} (HMM) and \textit{temporal evolution patterns} (DMD) invisible to raw article counts or ratios. Their contribution is primarily through enhanced interpretability and mechanistic understanding.

\textbf{HMM Detects Regime Transitions}: HMM features rank prominently in importance rankings:
    \begin{itemize}
    \item hmm\_ratio\_transition\_risk: 0.032 importance (5th highest feature in XGBoost Advanced).
    \item hmm\_ratio\_crisis\_prob: 0.025 importance (10th highest).
    \end{itemize}

What do these features capture? HMM detects \textit{latent regime transitions}---shifts in the underlying probabilistic structure of news narratives. For example, a district may have stable article volumes (50-60 articles/month) but undergo a regime shift from ``peaceful development'' to ``conflict-prone crisis'' narratives. HMM's hidden states (estimated via Baum-Welch algorithm with 3 states: stable, transitioning, crisis) identify when narrative regimes change even when raw article counts remain constant.

\textbf{When Hidden Variables Matter}: HMM provides unique value by detecting \textit{qualitative shifts in crisis narratives} that quantitative article counts miss. The hmm\_ratio\_transition\_risk feature ranks \#5 in importance (3.2\%), demonstrating that regime transition detection contributes meaningful signal for identifying when narrative regimes change even when raw article counts remain stable.

\textbf{DMD Identifies Crisis Evolution Patterns}: DMD (Dynamic Mode Decomposition) extracts temporal patterns (growth rates, oscillation frequencies, instability metrics) from 12-month rolling windows of news features. DMD achieves 83.1\% convergence rate across observations with sufficient temporal data (12 consecutive months). Mixed-effects models reveal dmd\_ratio\_crisis\_instability achieves the largest coefficient (+352.38) among all features, demonstrating that \textit{when DMD detects multi-category simultaneous spikes, it strongly signals complex emergencies}. While affecting <3\% of observations (by design$\times$these are rare crisis escalation events), DMD provides critical signal for detecting the most severe humanitarian catastrophes where multiple crisis drivers converge simultaneously.

\textbf{Contribution to Crisis Understanding}: HMM and DMD advance our understanding of \textit{how crises unfold}. HMM captures regime transitions (peaceful $\rightarrow$ violent narrative shifts) that raw article counts miss, with hmm\_ratio\_transition\_risk ranking \#5 in feature importance demonstrating substantial interpretability value. DMD identifies temporal modes characterising crisis escalation (positive growth rates) versus sustained intensity (near-zero eigenvalues), with dmd\_ratio\_crisis\_instability achieving the largest mixed-effects coefficient (+352.38) for detecting extreme complex emergencies. Together, these methods demonstrate that \textbf{latent variable approaches capture crisis evolution patterns that cross-sectional aggregations cannot}, justifying their inclusion for interpretable early warning systems where understanding \textit{why} a prediction changed matters as much as the prediction itself.

\subsection{RQ4 Answered: Two-Stage Framework Performance and Trade-Offs}

\textbf{Research Question}: Can a two-stage residual modelling approach effectively rescue crises missed by autoregressive baselines, and what are the precision-recall trade-offs of such a framework?

\textbf{Key Saves: 249 Rescued Crises (17.4\% Rescue Rate)}: The two-stage cascade framework successfully identifies 249 crises (out of 1,427 AR failures, 17.4\% rescue rate) that AR baseline missed but Stage 2 XGBoost Advanced correctly predicted. These \textbf{key saves} represent the framework's core humanitarian value---providing 8-month early warnings for crises that simple persistence models overlook.

\textbf{These are not just numbers---they are the cases that matter most.} The 249 key saves represent \textit{the hardest-to-predict crises} where spatio-temporal persistence breaks down---conflict-driven shocks in Sudan's Darfur region, economic collapse in Zimbabwe, complex emergencies in DRC's eastern provinces. These are precisely the crises where 8-month advance warning makes the difference between reactive disaster response and proactive humanitarian intervention. The cascade improvement from Recall=0.732 to 0.779 might appear as ``just'' a +4.7 percentage point gain in aggregate metrics, but \textbf{this obscures what operationally matters: 249 real crises affecting millions of people, now predicted 8 months in advance when they were previously invisible}. These early warnings enable pre-positioning of food stocks, negotiation of humanitarian access before violence escalates, and mobilisation of emergency funding before populations exhaust coping strategies. The cascade is not delivering modest statistical refinement---it is rescuing the most critical cases where persistence fails and where timely intervention saves lives.

Geographic distribution of key saves (from cascade\_optimised\_production results):
    \begin{itemize}
    \item Zimbabwe: 77 saves (30.9\% of total) --- Economic collapse, inflation crises, rapid IPC deteriorations in previously stable districts.
    \item Sudan: 59 saves (23.7\%) --- Conflict escalations, displacement-driven food insecurity in Darfur and Kordofan.
    \item Democratic Republic of the Congo: 40 saves (16.1\%) --- Complex emergencies, multi-causal crises (conflict + displacement + health).
    \item Nigeria: 27 saves (10.8\%) --- Boko Haram insurgency spillover, sudden market disruptions in Borno State.
    \item Other 14 countries: 46 saves (18.5\%) --- Dispersed across Kenya, Ethiopia, Mozambique, Mali, Malawi, Somalia.
    \end{itemize}

These three countries---Zimbabwe, Sudan, DRC---account for 70.7\% of all key saves despite representing only 16.7\% of total countries (3 out of 18). This geographic concentration reflects both \textit{high baseline crisis rates} (more opportunities for AR failures) and \textit{strong news coverage} (enabling informative signals).

    \begin{figure}[H]
        \centering
    \includegraphics[width=0.95\textwidth]{figures/ch05_discussion/ch05_cascade_breakthrough.pdf}
    \caption[Cascade Breakthrough: 249 Crises Rescued Where AR Failed]{
        \textbf{Cascade Breakthrough: News Features Rescue 249 Shock-Driven Crises Where Persistence Models Failed.}
        Flow diagram illustrating the two-stage cascade framework's success on the hardest-to-predict cases. Of 5,322 total crises, the AR baseline successfully predicted 3,895 (73.2\%)---these are \textit{easy cases} characterised by temporal persistence and spatial clustering. The AR baseline failed on 1,427 crises (26.8\%)---these are \textit{hard cases} characterised by rapid-onset shocks (conflict escalations, economic collapse, displacement crises) where ``yesterday predicts today'' breaks down. Stage 2 XGBoost Advanced (35 features: ratio, z-score, HMM, DMD, location) successfully rescued \textbf{249 of these hard cases} (17.4\% rescue rate), concentrated in three conflict/economic-driven contexts: Zimbabwe (77 saves, economic collapse), Sudan (59 saves, conflict escalation), DRC (40 saves, displacement shocks). These 249 key saves represent \textit{the breakthrough on cases that matter most}---crises invisible to temporal and spatial persistence, now predicted 8 months in advance, enabling preemptive food assistance, livelihood support, conflict-sensitive programming, and emergency funding mobilisation before populations exhaust coping strategies. The remaining 1,178 hard cases (82.6\%) remain difficult for both AR and news-based methods, indicating limits of GDELT English-language coverage for detecting subtle, slow-onset, or low-visibility crises in news-sparse regions (Madagascar, Niger, Uganda). \textbf{Key insight}: The cascade does not improve average performance (F1 decreases 0.732 $\times$ 0.668)---it strategically targets the \textit{hardest} cases where news signals provide genuine early warning value beyond autocorrelation. This is not statistical refinement---it is humanitarian impact where it matters most.
        \textit{Data: n=20,722 observations, 5,322 crises, h=8 months, 5-fold stratified spatial CV. }
    }
    \label{fig:ch5_cascade_breakthrough}
    \end{figure}

    \begin{figure}[H]
        \centering
    \includegraphics[width=0.95\textwidth]{figures/ch05_discussion/ch05_cascade_geographic_map.pdf}
    \caption[Geographic Distribution of 249 Cascade Rescues]{
        \textbf{Geographic Distribution of 249 Cascade Rescues: Where News Features Provided Genuine Early Warning Value.}
        District-level map showing the spatial distribution of key saves---cases where AR baseline predicted NO crisis (ar\_pred=0), actual outcome was CRISIS (y\_true=1), and cascade Stage 2 correctly predicted CRISIS (cascade\_pred=1). The 249 rescues span 118 districts across 10 countries but exhibit strong geographic concentration: 70.7\% occur in three countries highlighted in gold (Zimbabwe: 77 saves, Sudan: 59 saves, DRC: 40 saves). Circle size indicates number of key saves per district; colour intensity (yellow-orange-red) shows rescue frequency. This concentration reflects contexts where shock-driven crises (conflict escalations in Sudan, economic collapse in Zimbabwe, displacement in DRC) break temporal and spatial persistence patterns, enabling news features to provide marginal value beyond autocorrelation. Sparse coverage in East Africa (Kenya: 8 saves, Ethiopia: 6 saves) and zero saves in Madagascar/Niger reflect regions where spatial autoregressive features already capture regional climate patterns (droughts affect neighbours simultaneously) or where GDELT English-language coverage is insufficient. \textbf{Operational insight}: The map identifies where news-based cascade deployment provides humanitarian value (conflict/economic zones with high GDELT coverage) versus where expanding NLP text sources (social media, humanitarian reports, local-language news) is needed to address sparse English-language coverage.
        \textit{Data: 249 key saves across 118 districts, 10 countries. }
    }
    \label{fig:ch5_cascade_geographic_map}
    \end{figure}

\textbf{Precision-Recall Trade-Off}: The cascade framework prioritises recall over precision, reflecting humanitarian priorities where missing a crisis (false negative) is far more costly than false alarms (false positives):

    \begin{table}[htbp]
    \centering
\caption{AR Baseline vs Cascade Framework Performance}
    \begin{tabular}{lcc}
\toprule
\textbf{Metric} & \textbf{AR Baseline (h=8)} & \textbf{Cascade Framework} \\
\midrule
Precision & 0.732 & 0.585 (-14.7pp) \\
Recall & 0.732 & 0.779 (+4.7pp) \\
F1 Score & 0.732 & 0.668 (-6.4pp) \\
\midrule
True Positives & 3,895 & 4,144 (+249) \\
False Positives & 1,427 & 2,939 (+1,512) \\
False Negatives & 1,427 & 1,178 (-249) \\
True Negatives & 13,973 & 12,461 (-1,512) \\
\bottomrule
    \end{tabular}
    \end{table}

The cascade's 249 additional true positives (key saves) come at the cost of 1,512 additional false positives---a 6.1:1 trade-off ratio (1 rescued crisis costs 6.1 false alarms). Traditional ML metrics (F1 score decreases from 0.732 to 0.668) suggest this is unfavourable, but humanitarian cost-sensitive evaluation tells a different story.

\textbf{Cost-Sensitive Analysis}: Assigning asymmetric costs (missing a crisis = 10$\times$ worse than false alarm, standard humanitarian ratio reflecting intervention costs vs lives at risk):

    \begin{itemize}
    \item AR Baseline cost: $10 \times 1,427$ (FN) + $1 \times 1,427$ (FP) = 15,697 units
    \item Cascade cost: $10 \times 1,178$ (FN) + $1 \times 2,939$ (FP) = 14,719 units
    \item \textbf{Improvement}: -978 cost units (-6.2\% reduction)
    \end{itemize}

At 10:1 FN:FP weighting, the cascade provides net benefit despite precision loss. At lower weightings (e.g., 5:1), AR baseline remains preferable. This highlights a critical operational decision: \textit{the optimal framework depends on organisational cost tolerance for false alarms versus missed crises}.

\textbf{Opportunities for Enhanced Signal Extraction---The Remaining 1,178 Cases}: Beyond the 249 rescued crises, 1,178 of the 1,427 cases requiring complementary signals (82.6\%) represent opportunities for advanced signal extraction techniques. Figure \ref{fig:ch5_cascade_failures} analyses why these cases remain difficult to predict.

    \begin{figure}[H]
        \centering
    \includegraphics[width=0.95\textwidth]{figures/ch05_discussion/ch05_cascade_failures_analysis.pdf}
    \caption[Cascade Failures Analysis: Why 1,178 Cases Still Missed]{
        \textbf{Cascade Failures Analysis: Why 1,178 Crises (82.6\%) Remain Unpredictable After News-Based Intervention.}
        Four-panel visualisation analysing the 1,178 still-missed cases where both AR baseline and cascade failed to predict crisis. Panel A (Geographic Distribution) shows top 3 countries: Kenya (234 still-missed, 3\% rescued), Zimbabwe (188 still-missed, 29\% rescued), Sudan (171 still-missed, 26\% rescued)---different from key saves concentration, indicating news features work better in some contexts. Panel B (News Coverage Comparison) reveals the critical finding: still-missed cases have \textbf{64\% less news coverage} (median 74 articles/month) compared to key saves (median 121 articles/month)---news density determines whether cascade can rescue AR failures. Panel C (Feature Deficiency Analysis) shows still-missed cases have 19\% fewer unique sources, lower category entropy, and weaker conflict signals---insufficient feature richness prevents model discrimination. Panel D (Summary) introduces the \textbf{news deserts hypothesis}: Remote pastoral areas (Kenya Northern, Zimbabwe rural districts, Sudan periphery) lack sufficient GDELT English-language coverage for news-based features to add predictive value beyond AR baseline. The 82.6\% of AR failures that cascade cannot rescue represent \textit{fundamentally different crisis contexts}---slow-onset malnutrition, localized drought impacts, chronic vulnerability---requiring expanded NLP text sources (social media monitoring, humanitarian situation reports from OCHA/UNHCR/WFP, community radio transcripts, local-language news in Swahili/Hausa/Amharic) to complement sparse English-language news signals. \textbf{Key insight}: Cascade success depends on news density---works in high-coverage conflict zones (Sudan Darfur, DRC Kivu), fails in news-sparse pastoral regions (Kenya Turkana, Niger Diffa) where diversifying text corpora is essential.
        \textit{Data: 1,178 still-missed cases, 249 key saves, compared across geographic distribution, news coverage (article\_count, unique\_sources), and feature strength (category\_entropy, conflict\_ratio). }
    }
    \label{fig:ch5_cascade_failures}
    \end{figure}

Two primary opportunity areas emerge:

    \begin{enumerate}
    \item \textbf{Coverage density variation}: Cases occur across districts with varying GDELT coverage patterns (<200 to >2,000 articles/year). Niger (67 cases, 0 key saves with current methods) demonstrates an opportunity zone: regions where limited English-language coverage suggests expanding NLP text sources (multilingual news processing, local radio transcripts, social media monitoring, humanitarian field reports) could provide complementary information channels.

    \item \textbf{Different crisis manifestation patterns}: Some crises exhibit distinct temporal signatures (gradual environmental degradation, chronic malnutrition) that may benefit from different NLP analytical approaches---long-term narrative trend analysis from humanitarian reports, agricultural bulletins, and development agency assessments that capture slow-onset processes beyond rapid-event news coverage.
    \end{enumerate}

This 82.6\% represents substantial opportunity for methodological enhancement through advanced NLP techniques. Future work directions include: (1) transformer-based semantic understanding (BERT/RoBERTa fine-tuned on crisis corpora), (2) multilingual models capturing French/Arabic/Swahili regional news expanding beyond English-only GDELT, (3) social media text mining for additional real-time crisis signals, (4) automated event extraction identifying specific crisis triggers from narrative text---demonstrating that current news themes features establish a foundation for more sophisticated text analysis approaches.

\subsubsection{The News Deserts Hypothesis: A Fundamental Constraint on News-Based Early Warning}

\textbf{Additional Insights}: Cascade failure analysis (Figure \ref{fig:ch5_cascade_failures}, Panel B) reveals a systematic structural constraint: the 1,178 still-missed cases exhibit \textbf{64\% less news coverage} than the 249 successfully rescued cases (median 74 vs 121 articles/month, p<0.001). This coverage deficiency is not random---it concentrates in specific geographic contexts (Figure \ref{fig:ch5_cascade_failures_map}):

    \begin{figure}[H]
        \centering
    \includegraphics[width=0.95\textwidth]{figures/ch05_discussion/ch05_cascade_failures_map.pdf}
    \caption[Geographic Distribution of Cascade Failures: News Deserts]{
        \textbf{Geographic Distribution of 1,178 Cascade Failures: News Deserts Where Insufficient Coverage Prevents Rescue.}
        Map showing district-level distribution of still-missed cases across Africa. Point size indicates failure density per district; colour indicates median news coverage (red/pink=low coverage news deserts, purple=better coverage). Still-missed cases exhibit systematic news coverage deficiency (median 79 articles/month) compared to 249 rescued cases (median 121 articles/month, 53\% more).
        Within-country heterogeneity analysis reveals the same countries show both cascade rescues and failures at district level. Zimbabwe has 77 key saves (Figure \ref{fig:ch4_key_saves_map}) but 647 failures shown here; Sudan has 59 saves but 420 failures; Kenya has 17 saves but 722 failures. This pattern demonstrates district-level news coverage heterogeneity within countries. Well-covered districts (capitals like Harare/Khartoum, conflict zones like Eastern DRC, economically significant areas) enable cascade rescue (purple bubbles); news desert districts (remote pastoral areas like Kenya Northern/Turkana, peripheral regions like Zimbabwe rural districts) lack sufficient media coverage for news-based features to add value (red/pink bubbles). Purple districts indicate sufficient coverage but cascade failed for other reasons (complex dynamics, data quality); red/pink districts indicate insufficient coverage (true news deserts where prediction is fundamentally impossible without media presence).
        Failures concentrate in three countries (red borders): Kenya (722, 24.7\%), Zimbabwe (647, 22.1\%), Sudan (420, 14.3\%). Demonstrates fundamental constraint: you cannot predict what is not reported. Future NLP enhancements must expand text corpora (social media, community radio, humanitarian reports, multilingual sources) to address news deserts.
        \textit{n=1,178 still-missed cases across 459 districts in 18 countries, h=8 months.}
    }
    \label{fig:ch5_cascade_failures_map}
    \end{figure}

    \begin{itemize}
    \item \textbf{Remote pastoral zones}: Kenya Northern (Turkana, Marsabit), Zimbabwe rural districts, Niger Diffa---regions with sparse media presence where crises unfold beyond journalistic reach.
    \item \textbf{Slow-onset malnutrition}: Chronic vulnerability contexts lacking acute "newsworthy" events (conflict, displacement, coups) that attract media attention.
    \item \textbf{Peripheral regions}: Districts far from capital cities and conflict zones where international news agencies maintain limited presence.
    \end{itemize}

\textbf{Why This Matters}: The news deserts hypothesis reveals a fundamental constraint on news-based early warning systems: \textit{you cannot predict what is not reported}. Unlike satellite imagery (which covers all geographic areas uniformly) or household surveys (which can be targeted to underreported regions), news media coverage is inherently uneven---concentrated in politically important, conflict-affected, and economically significant areas while neglecting remote pastoral zones and chronically vulnerable regions.

The 249 key saves concentrate in news-dense conflict zones (70.7\% in Sudan/Zimbabwe/DRC) precisely because these contexts generate the news coverage that enables dynamic feature extraction. The 1,178 still-missed cases concentrate in news-sparse regions where insufficient article density prevents robust feature engineering---HMM convergence requires temporal depth (48+ months of coverage), DMD mode extraction requires sufficient variation, and ratio features require stable compositional baselines.

\textbf{NLP Recommendations for Addressing News Deserts}: To expand early warning coverage to underreported regions, future NLP systems must diversify text corpora beyond traditional English-language news:

    \begin{enumerate}
    \item \textbf{Social media monitoring}: Twitter/X posts, Facebook community pages, WhatsApp group analysis (where available with privacy protections) provide grassroots crisis signals from affected populations directly, bypassing traditional news gatekeepers.

    \item \textbf{Community radio transcripts}: Local-language broadcasts in Swahili, Hausa, Amharic, Somali, French, Arabic reach remote audiences and cover localized crises that international media misses. Automated speech-to-text with language-specific models enables text mining from audio archives.

    \item \textbf{Humanitarian situation reports}: OCHA (UN Office for the Coordination of Humanitarian Affairs), UNHCR (refugee/displacement reports), WFP (food security assessments) produce regular text-based crisis documentation with systematic geographic coverage. Mining these reports complements news-based features.

    \item \textbf{Multilingual news sources}: Expanding beyond GDELT's English-language corpus to French-language news (covering Francophone Africa: Niger, Mali, Burkina Faso, DRC), Arabic-language sources (Sudan, Somalia, Mauritania), and Portuguese sources (Mozambique, Angola) captures regional perspectives invisible in English media.

    \item \textbf{Targeted collection strategies}: For persistently underreported regions (Niger, Madagascar, Uganda rural districts), proactive text collection partnerships with local journalists, NGO field reports, and regional news aggregators can fill coverage gaps.
    \end{enumerate}

The news deserts hypothesis transforms how we understand cascade limitations: the 82.6\% of still-missed cases are not primarily a modelling failure (better algorithms extracting more signal from existing text) but a \textbf{data availability constraint} (insufficient text exists to extract signal from). Addressing this requires expanding NLP data sources, not just refining feature engineering. This insight fundamentally reshapes deployment strategy: deploy news-based cascades in high-coverage contexts (Sudan/Zimbabwe/DRC conflict zones) while investing in alternative text corpora collection (social media, radio, humanitarian reports) for low-coverage contexts (pastoral zones, peripheral regions).

\subsection{RQ5 Answered: Geographic Heterogeneity---Where News Matters Most}

\textbf{Research Question}: Are news-based features equally valuable across all geographic contexts, or do certain countries and crisis types benefit more from dynamic news signals than others?

\textbf{Strong Heterogeneity Observed}: Performance varies dramatically across countries:

    \begin{itemize}
    \item \textbf{Best performers}: Sudan (AUC 0.682), Uganda (0.679), Kenya (0.637)---all have high news density and established crisis patterns that news features can learn.
    \item \textbf{Contexts with distinct dynamics}: Niger (AUC 0.068), Ethiopia (0.417), Mozambique (0.515)---contexts with different coverage densities and crisis patterns requiring tailored analytical approaches.
    \item \textbf{Range}: 0.614 AUC difference (10$\times$ performance variation)---news features provide strong value in specific countries while requiring complementary approaches in others.
    \end{itemize}

    \begin{figure}[H]
        \centering
    \includegraphics[width=0.95\textwidth]{figures/ch05_discussion/fig1_delta_auc_country_rankings.pdf}
    \caption[Geographic Heterogeneity in News Value]{Geographic Heterogeneity in News Value: Delta-AUC by Country. Marginal performance gain (cascade balanced accuracy minus AR baseline) reveals dramatic variation and paradoxical pattern: most countries show negative Delta-AUC despite providing key saves. High Benefit countries (Zimbabwe -0.017, Sudan -0.068, DRC -0.084 most negative, Nigeria -0.063, n=7) achieve substantial key saves (77, 59, 40, 27) while accepting precision loss. Somalia (+0.0013) shows rare positive Delta-AUC. AR Superior countries (Madagascar -0.079, Malawi -0.025, n=2) demonstrate baseline sufficiency---negative Delta-AUC without compensating key saves. Statistical validation (Kruskal-Wallis H=7.82, p=0.020) confirms significant heterogeneity---news value measured by humanitarian impact (key saves), not aggregate metrics (Delta-AUC).}
    \label{fig:delta_auc_country}
    \end{figure}

This heterogeneity is not random noise---it systematically correlates with three factors:

\textbf{1. News Coverage Density}: country\_data\_density (articles per district per year) ranks highest in tree-based importance (0.133 split frequency). High-coverage countries produce more training observations, enabling models to learn reliable patterns. Low-coverage countries suffer from sparse data, producing unstable estimates. Note: Tree-based importance measures stratification utility; SHAP reveals this feature contributes only 2.6\% marginal attribution as part of location metadata serving as geographic context infrastructure.

\textbf{2. Baseline Conflict Intensity}: country\_baseline\_conflict ranks second in importance (0.093). Chronic conflict zones (Sudan, DRC, Nigeria) develop predictable crisis patterns tied to conflict escalations, while peaceful countries (Madagascar, Malawi) have sporadic crises driven by diverse causes that news features struggle to capture.

\textbf{3. Crisis Type}: Mixed-effects random effects quantify country-specific deviations from global patterns:

    \begin{itemize}
    \item \textbf{Positive random effects} (news features help more than average):
    \begin{itemize}
        \item Somalia: +3.70 (highest)---conflict-driven famines with strong news signals.
        \item Zimbabwe: +2.67---economic crises with extensive international coverage.
        \item Sudan: +2.24---chronic conflict with predictable escalation patterns.
    \end{itemize}

    \item \textbf{Negative random effects} (news features help less than average):
    \begin{itemize}
        \item Madagascar: -4.56 (lowest)---climate/cyclone-driven crises with distinct temporal patterns where English-language news coverage is sparse, requiring expanded local-language text sources (Malagasy-language news, regional bulletins, cyclone impact reports).
        \item Uganda: -3.86---food security contexts with localized patterns benefiting from targeted approaches.
        \item Kenya: -0.35---pastoral mobility reduces spatial signal strength, news adds little.
    \end{itemize}

    \item \textbf{Range}: 8.26 log-odds points---massive heterogeneity suggesting country-specific models may outperform pooled global models.
    \end{itemize}

\textbf{Crisis Type Variation---Why Zimbabwe, Sudan, and DRC Dominate Key Saves}:

The 70.7\% concentration of key saves in three countries is not a data artifact---it reflects genuine differences in crisis dynamics:

    \begin{itemize}
    \item \textbf{Conflict-driven crises} (Sudan, DRC, Nigeria): Rapid onset, strong news signals (violence reporting), sudden IPC deteriorations. AR baseline assumes gradual transitions (Lt captures slow changes), so conflict shocks break persistence assumptions. News features capturing conflict escalations provide genuine early warning.

    \item \textbf{Economic crises} (Zimbabwe): Structural transitions (inflation, currency devaluation, market shifts) produce sudden IPC changes with distinct patterns. News coverage of economic policies, inflation rates, and market dynamics provides complementary leading indicators. AR models optimised for temporal persistence benefit from news-based augmentation during rapid economic transitions.

    \item \textbf{Climate-driven crises} (Kenya, Ethiopia pastoral zones): Droughts and floods affect large regions simultaneously, producing high spatial autocorrelation. Ls (spatial autoregressive features) already captures regional patterns, so news features add minimal marginal value. Key saves are sparse in these regions (Kenya: 8 saves, Ethiopia: 6 saves) because AR baseline already performs well via spatial signals.
    \end{itemize}

    \begin{figure}[H]
        \centering
    \includegraphics[width=0.95\textwidth]{figures/ch05_discussion/fig2_key_saves_concentration.pdf}
    \caption[Geographic Concentration of Cascade Impact]{Geographic Concentration of Cascade Impact: 70.7\% of Key Saves in Three Countries. Zimbabwe (77 saves, 30.9\%), Sudan (59, 23.7\%), and DRC (40, 16.1\%) dominate humanitarian impact. This concentration reflects genuine crisis dynamics---conflict zones and economic collapses where AR fails and news provides marginal value. Long tail distribution: 15 remaining countries contribute 73 saves (29.3\%), suggesting selective deployment strategy over universal application.}
    \label{fig:key_saves_concentration}
    \end{figure}

\textbf{Implications for Deployment}: Selective deployment is necessary:

    \begin{enumerate}
    \item \textbf{Use cascade framework} in: Sudan, Zimbabwe, DRC, Nigeria (conflict/economic-driven crises, high coverage, demonstrated key saves).
    \item \textbf{Use AR baseline only} in: Kenya, Ethiopia pastoral zones, Madagascar, Malawi (climate-driven or sparse coverage, minimal cascade improvement).
    \item \textbf{Uncertain benefit} in: Mali, Mozambique, Somalia (moderate key saves, cost-benefit depends on operational resources).
    \end{enumerate}

Geographic heterogeneity is not a limitation to be overcome---it is a signal to be respected. News features provide value where crisis dynamics generate informative text signals (conflict, economic shocks, displacement) but not universally. One-size-fits-all deployment would waste resources applying complex models where simple AR baselines suffice.

\textbf{Country-Specific Theme Patterns: Which News Themes Drive Predictions Where?}

Beyond identifying \textit{where} news matters (geographic heterogeneity in key saves), we can investigate \textit{which themes} drive predictions in each country by analysing observation-level SHAP values (n=23,039 across 13 countries). Aggregating SHAP importance for both ratio and z-score features by theme reveals country-specific signatures:

\textbf{Zimbabwe} (77 key saves): Humanitarian (13.4\%), Other (13.0\%), Weather (11.5\%) dominate. This aligns with economic collapse context---humanitarian crisis reporting, general instability coverage, and climate shocks (2019 Cyclone Idai, 2022-2023 drought) drive predictions. Weather ranks 3rd here but 8th globally (9.4\%), confirming Zimbabwe's vulnerability to climate extremes compounding economic fragility.

\textbf{Sudan} (59 key saves): Governance (14.8\%), Conflict (14.6\%), Humanitarian (13.4\%) lead. This signature reflects April 2023 conflict escalation (RSF vs SAF), state collapse, and resulting humanitarian emergency. Conflict ranks 2nd in Sudan but 4th globally (11.3\%), demonstrating context-specific amplification: rapid-onset violence shocks in Sudan contrast with chronic low-intensity conflicts elsewhere.

\textbf{DRC} (40 key saves): Other (14.3\%), Humanitarian (12.9\%), Displacement (12.2\%) characterise complex emergency dynamics. Displacement ranks 3rd in DRC but 7th globally (10.0\%), capturing M23 resurgence (2022-2023), internal displacement flows, and protracted refugee crises absent in stable countries.

\textbf{Global average} (13 countries): Governance (13.0\%) and Other (13.0\%) lead, followed by Humanitarian (12.6\%) and Conflict (11.3\%). This relatively flat distribution (9.2\% to 13.0\%, 3.8 percentage point range) suggests no single theme universally dominates---importance depends on country-specific crisis dynamics.

\textbf{Key insight}: Theme importance rankings shift substantially by country. Zimbabwe prioritises Weather (11.5\% vs 9.4\% global), Sudan prioritises Conflict (14.6\% vs 11.3\% global), DRC prioritises Displacement (12.2\% vs 10.0\% global). This heterogeneity reinforces selective deployment logic: not only \textit{where} to use news (Zimbabwe/Sudan/DRC) but \textit{which themes} matter most in each context. Universal theme weighting would miss country-specific signals.

\textbf{Operational implications for theme-aware deployment.} Country-specific theme signatures enable more nuanced monitoring strategies beyond binary ``use news / don't use news'' decisions. For countries with demonstrated cascade value (Zimbabwe 77 saves, Sudan 59, DRC 40), practitioners can prioritise real-time monitoring of country-specific themes showing elevated SHAP importance: weather alerts for Zimbabwe (cyclones, droughts), conflict bulletins for Sudan (violence escalations, ceasefire violations), displacement tracking for DRC (refugee flows, IDP movements). This theme-targeted surveillance reduces information overload---humanitarian analysts need not track all nine themes equally across all contexts---while maintaining sensitivity to the specific shock types that drive predictions in each country. The relatively flat global distribution (3.8pp range, 1.4$\times$ max/min ratio) confirms the absence of a universal ``magic bullet'' theme: effective early warning requires context-aware thematic prioritisation based on country-specific crisis dynamics revealed through SHAP analysis.

    \begin{figure}[H]
        \centering
    \includegraphics[width=0.95\textwidth]{figures/ch05_discussion/fig5_country_theme_heatmap.pdf}
    \caption[Country-Specific News Theme Importance Heatmap]{Country-Specific News Theme Importance Heatmap. SHAP-based analysis (n=23,039 observations, 13 countries) reveals which themes drive cascade predictions in each context. Countries sorted by key saves (Zimbabwe, Sudan, DRC top); themes sorted by global importance (Governance 13.0\%, Other 13.0\%, Humanitarian 12.6\%). Zimbabwe shows elevated Weather importance (11.5\% vs 9.4\% global); Sudan shows elevated Conflict (14.6\% vs 11.3\%); DRC shows elevated Displacement (12.2\% vs 10.0\%). Relatively flat global distribution (9.2-13.0\%, 3.8pp range) indicates no universal dominant theme---importance varies by country-specific crisis dynamics. Data source: Mean absolute SHAP values for ratio + z-score features aggregated by theme category.}
    \label{fig:country_theme_heatmap}
    \end{figure}

    \begin{figure}[H]
        \centering
    \includegraphics[width=0.95\textwidth]{figures/ch05_discussion/fig6_top3_theme_comparison.pdf}
    \caption[Theme Signatures for Top 3 Countries]{Theme Signatures for Top 3 Countries by Key Saves. Direct comparison of theme importance in Zimbabwe (77 saves), Sudan (59), and DRC (40). Zimbabwe: Humanitarian-Weather-focused (reflecting economic collapse + climate shocks). Sudan: Governance-Conflict-driven (reflecting April 2023 state collapse). DRC: Humanitarian-Displacement-dominated (reflecting complex emergency with M23 resurgence). Bars sorted by importance within each country; value labels show exact percentages. Distinct signatures confirm context-specific news utilisation: models learn different thematic patterns in different crisis types.}
    \label{fig:top3_theme_comparison}
    \end{figure}

\subsubsection{Geographic Patterns in News Theme Importance}

While the preceding heatmap (Figure~\ref{fig:country_theme_heatmap}) and bar charts (Figure~\ref{fig:top3_theme_comparison}) reveal \textit{which} themes matter \textit{where}, geographic visualisation makes spatial patterns immediately apparent and reveals clusters of thematic similarity. We visualise theme-geography relationships through two complementary map perspectives that illuminate both absolute dominance and context-specific amplification.

\textbf{Dominant themes reveal regional crisis typologies.} Figure~\ref{fig:country_theme_map} maps the theme with highest absolute SHAP importance in each country, exposing clear geographic clustering. Governance dominates across a northern belt (Sudan 14.8\%, Nigeria, Ethiopia) and southern arc (Malawi, Madagascar), reflecting shared political fragility and state capacity challenges in these contexts. The Sahel and East African political transitions produce governance-heavy news coverage that cascade models leverage for predictions. In contrast, Other dominates in Central/West Africa (DRC 14.3\%, Mozambique, Mali, Niger), indicating complex multi-faceted crises where no single thematic category captures the heterogeneous news landscape---conflict, displacement, health, and food security co-occur and intermingle in reporting.

    \begin{figure}[H]
        \centering
    \includegraphics[width=0.95\textwidth]{figures/ch05_discussion/fig7_country_theme_map.pdf}
    \caption[Geographic Distribution of Dominant News Themes]{Geographic Distribution of Dominant News Themes Across Africa. SHAP-based choropleth map (n=23,039 observations, 13 countries) showing which theme contributes most to cascade predictions in each country. Zimbabwe: Humanitarian dominant (13.4\%, reflecting economic collapse + hyperinflation). Sudan: Governance dominant (14.8\%, reflecting April 2023 state collapse). DRC: Other dominant (14.3\%, reflecting complex multi-faceted emergency). Governance dominant in 5/13 countries (Sudan, Nigeria, Ethiopia, Malawi, Madagascar); Other dominant in 4/13 (DRC, Mozambique, Mali, Niger). Red borders highlight top 3 by key saves (Zimbabwe 77, Sudan 59, DRC 40 = 70.7\% of total). All 13 countries labelled with dominant theme and key saves count. Relatively flat global theme distribution (9.2-13.0\%, 3.8pp range) confirms no universal dominant theme---news value depends on country-specific crisis dynamics. Map demonstrates not just \textit{where} news matters (geographic concentration) but \textit{which themes} dominate in each context.}
    \label{fig:country_theme_map}
    \end{figure}

Humanitarian themes appear distinctly in Zimbabwe (13.4\%), reflecting the unique hyperinflation crisis where economic collapse reporting (currency devaluation, market failures, humanitarian appeals) drives predictions more than traditional conflict or weather signals. Health concentrates in East Africa (Kenya, Somalia), potentially reflecting disease outbreaks and medical infrastructure challenges that compound food insecurity in these regions. This geographic structure demonstrates that theme importance is not randomly distributed but follows regional crisis typologies---countries experiencing similar types of shocks rely on similar thematic signals for prediction.

\textbf{Theme elevations reveal shock-specific amplification.} Figure~\ref{fig:theme_elevation_map} provides a complementary perspective by mapping which theme is \textit{most elevated above global average}---revealing context-specific amplification rather than absolute dominance. This elevation metric (local percentage minus global average) identifies which shock types are overrepresented in each country's news landscape relative to the continental baseline. Notably, this produces different geographic patterns than dominant theme mapping: Conflict elevates most in Sudan (+3.3pp, 14.6\% vs 11.3\% global) and Mali (violence hotspots where civil war coverage far exceeds typical conflict reporting levels); Weather elevates in Zimbabwe/Ethiopia/Malawi/Madagascar (+2.1pp to +3.0pp range), all climate-vulnerable agricultural economies where drought and cyclone coverage dominates local news cycles.

    \begin{figure}[H]
        \centering
    \includegraphics[width=0.95\textwidth]{figures/ch05_discussion/fig7_theme_elevation_map.pdf}
    \caption[Theme Elevations by Country]{Theme Elevations: Maximum Deviation from Global Average by Country. Complementary view showing which theme is \textit{most elevated} (not most dominant) in each country---largest positive deviation from global average importance. Zimbabwe: Weather +2.1pp (11.5\% vs 9.4\% global)---drought cycles compound economic fragility. Sudan: Conflict +3.3pp (14.6\% vs 11.3\%)---April 2023 civil war escalation produces maximum elevation. DRC: Displacement +2.2pp (12.2\% vs 10.0\%)---M23 resurgence and North Kivu flows. Kenya: Food Security +3.5pp (12.8\% vs 9.2\%)---harvest failures amplify baseline vulnerability. Somalia: Health +5.8pp (16.5\% vs 10.7\%)---disease burden compounds food insecurity (highest elevation observed). Elevation = Local \% - Global \%, revealing context-specific amplification. Red borders mark top 3 key saves countries. Key distinction from Fig.~\ref{fig:country_theme_map}: Dominant theme shows absolute highest \% (e.g., Zimbabwe Humanitarian 13.4\%); elevated theme shows maximum relative deviation (e.g., Zimbabwe Weather +2.1pp despite ranking 3rd locally). Both perspectives inform selective deployment---dominant themes show what drives predictions most; elevated themes show what differs from global patterns.}
    \label{fig:theme_elevation_map}
    \end{figure}

Displacement elevates maximally in DRC (+2.2pp, 12.2\% vs 10.0\% global), the M23 resurgence epicenter where refugee flows and IDP movements generate concentrated coverage. Food Security elevates in Kenya (+3.5pp) and Nigeria, harvest volatility regions where crop failure reporting exceeds continental norms. Somalia shows the highest elevation observed for any theme: Health at +5.8pp (16.5\% vs 10.7\% global), reflecting how disease burden (cholera, measles, malnutrition-related illnesses) compounds food insecurity and generates disproportionate medical emergency coverage in the Horn of Africa.

The distinction between dominant and elevated themes carries critical operational implications. Dominant themes (Figure~\ref{fig:country_theme_map}) show what drives predictions most in absolute terms, guiding where to allocate monitoring resources---humanitarian analysts should track governance signals in Sudan, humanitarian/economic signals in Zimbabwe, displacement in DRC. Elevated themes (Figure~\ref{fig:theme_elevation_map}) show what differs from global patterns, identifying which shock types require context-specific surveillance thresholds---weather alerts in Zimbabwe should trigger earlier than the global average given the +2.1pp elevation, while conflict monitoring in Sudan requires heightened sensitivity given +3.3pp elevation above baseline.

\textbf{Operational implications for theme-aware deployment.} The geographic concentration of elevated themes enables targeted monitoring strategies beyond binary ``use news / don't use news'' decisions. For countries with demonstrated cascade value (Zimbabwe 77 saves, Sudan 59, DRC 40), humanitarian organisations can prioritise real-time monitoring of country-specific elevated themes: (1) Weather monitoring systems for southern/eastern agricultural zones (Zimbabwe +2.1pp, Malawi, Kenya, Ethiopia, Madagascar) where climate shocks produce largest deviations from continental baseline; (2) Conflict early-warning systems for Sahel/Sudan corridor (Sudan +3.3pp, Mali, Niger) where violence spikes exceed global conflict patterns; (3) Displacement tracking for Great Lakes region (DRC +2.2pp, Uganda) where population movements dominate local news; (4) Health surveillance in East Africa (Somalia +5.8pp, Kenya) where disease burden compounds food insecurity. This theme-geography mapping enables customized data collection pipelines and alert thresholds by region rather than deploying uniform global monitoring infrastructure, reducing information overload while maintaining sensitivity to context-specific shock types.

\subsubsection{The Deployment Paradox: High Impact with Negative Performance Metrics}

Geographic heterogeneity analysis exposes a counterintuitive pattern that challenges conventional model evaluation: countries with the highest key saves (Zimbabwe 77, Sudan 59, DRC 40) simultaneously exhibit the most negative Delta-AUC values (-0.017, -0.068, -0.084 respectively). This apparent contradiction demands resolution: how can cascade rescue the most crises while degrading overall balanced accuracy? The answer lies in understanding what these metrics measure and which operational priorities they serve.

\textbf{Visualising the key saves--Delta-AUC paradox.} Figure~\ref{fig:news_value_scatter} maps this paradoxical relationship through a scatter plot positioning countries by key saves (y-axis, humanitarian impact) versus Delta-AUC (x-axis, model performance change). The visualisation immediately reveals a moderate negative correlation (Spearman $\rho$ = -0.648, p = 0.0036)---statistically significant evidence that countries contributing most to humanitarian impact show the worst aggregate performance metrics. The top 3 countries (green bubbles, upper-left quadrant) achieve 70.7\% of total cascade rescues despite posting the most negative Delta-AUC scores, forming a distinct cluster separated from the remaining countries.

    \begin{figure}[H]
        \centering
    \includegraphics[width=0.95\textwidth]{figures/ch05_discussion/fig3_news_value_scatter.pdf}
    \caption[Key Saves vs Delta-AUC Relationship]{Statistical Validation of Geographic Heterogeneity: Key Saves vs Delta-AUC Relationship. Scatter plot reveals moderate negative correlation (Spearman $\rho$=-0.648, p=0.0036)---countries with most key saves (Zimbabwe 77, Sudan 59, DRC 40, green bubbles) show negative Delta-AUC due to precision-recall trade-off. This paradox is not contradiction but confirmation: cascade rescues the hardest cases (high key saves) while accepting more false alarms (negative overall Delta-AUC). Top 3 countries form distinct cluster in upper-left quadrant. News value categories demonstrated through colour coding: High Benefit (green) includes countries with 15+ key saves prioritising humanitarian impact over balanced accuracy; Minimal Benefit (gray) shows minimal improvement with <10 saves; AR Superior (blue) indicates contexts where baseline persistence suffices. Bubble size represents total crisis count in country, showing that benefit depends on crisis type (rapid shocks) not frequency. Statistical significance (p=0.0036) confirms geographic heterogeneity is systematic, not random variation. Evidence-based classification guides deployment: deploy cascade where key saves justify false alarms, not where Delta-AUC maximised.}
    \label{fig:news_value_scatter}
    \end{figure}

This negative correlation emerges from precision-recall dynamics inherent to targeting AR failures. Countries with highest key saves exhibit the highest concentrations of rapid-onset shocks---exactly the cases AR baseline misses and cascade targets. Sudan (59 saves, -0.068 Delta-AUC) exemplifies this mechanism: the April 2023 conflict escalation produced sudden state collapse that AR's persistence assumption failed to anticipate, but news governance/conflict signals detected 8 months in advance. Rescuing these hardest cases requires relaxing decision thresholds to maximise recall, which necessarily increases false alarms (lower precision), degrading balanced accuracy. The paradox resolves when we recognise that Delta-AUC measures average performance across all predictions, while key saves measure success on the subset of predictions that matter most for humanitarian intervention.

The scatter plot colour-codes countries into three deployment categories based on the balance between humanitarian impact and performance cost. High Benefit countries (green: Zimbabwe/Sudan/DRC/Nigeria/Mozambique/Mali) show high key saves (15-77 rescues) despite negative Delta-AUC, justifying cascade deployment where saving lives outweighs false alarm costs. Minimal Benefit countries (gray: Kenya/Ethiopia and others) show near-zero Delta-AUC and minimal saves (<10 rescues), indicating cascade adds no value and AR baseline suffices. AR Superior countries (blue) show negative Delta-AUC without meaningful key saves, indicating persistence-dominated contexts where news features inject noise rather than signal.

Bubble sizes encode total crisis counts per country, revealing that High Benefit countries do not necessarily have the highest crisis rates---Kenya shows the largest bubble but minimal benefit. This confirms that cascade value depends on crisis \textit{type} (rapid shocks vs gradual deterioration) rather than crisis \textit{frequency}, validating the central thesis: news features provide marginal value selectively for shock-driven crises breaking autocorrelation, not universally across all food insecurity contexts.

\textbf{Multi-metric classification for deployment decisions.} While the scatter plot establishes the paradox, deployment decisions in humanitarian early-warning systems require evaluating countries across multiple performance dimensions simultaneously. A single metric provides insufficient evidence: high key saves without context could reflect high baseline crisis rates rather than genuine cascade value; negative Delta-AUC without examining rescue rates could lead to premature rejection of life-saving interventions. Figure~\ref{fig:cascade_benefit_matrix} addresses this gap through a comprehensive heatmap cross-classifying the top 12 countries by key saves across three critical metrics: Delta-AUC (overall model performance), Rescue Rate (percentage of AR failures rescued), and Recall Gain (absolute percentage point improvement).

    \begin{figure}[H]
        \centering
    \includegraphics[width=0.95\textwidth]{figures/ch05_discussion/fig4_cascade_benefit_matrix.pdf}
    \caption[Cascade Benefit Matrix for Top 12 Countries]{Cascade Benefit Matrix: Multi-Metric Performance Heatmap for Top 12 Countries. Normalised scores (0=worst, 1=best) across three dimensions reveal deployment paradox: High Benefit countries (first 6 columns) show negative Delta-AUC (red/orange in row 1) yet high Rescue Rates (green in row 2) and substantial Recall Gains (green in row 3). Zimbabwe: -0.017 Delta-AUC but 29.1\% rescue rate, +20.4pp recall gain. DRC: -0.084 Delta-AUC (worst) but 48.2\% rescue rate (second-highest), +14.2pp gain. Minimal Benefit countries (Kenya, Ethiopia, Malawi) show near-zero Delta-AUC degradation (green) but negligible rescue rates <5\% (red), confirming AR baseline sufficiency. Somalia uniquely demonstrates positive Delta-AUC (+0.001) with 18.2\% rescue rate, suggesting rare alignment of shock-driven crises and sufficient news density. Chad/Niger show 0.0\% rescue (dark red), definitively demonstrating news desert failure. Color intensity represents normalised score within each metric; actual values shown in cells. Matrix operationalizes selective deployment: prioritise countries with rescue rate >15\% and recall gain >+3pp (first 6 columns) despite negative Delta-AUC; avoid countries with rescue rate <5\% (last 6 columns) regardless of Delta-AUC. Classification balances humanitarian impact (lives saved) against aggregate accuracy, embodying 10:1 FN:FP cost weighting appropriate for early-warning systems.}
    \label{fig:cascade_benefit_matrix}
    \end{figure}

The benefit matrix reveals nuanced patterns invisible in univariate analysis. High Benefit countries (first 6 columns: Zimbabwe, Sudan, DRC, Nigeria, Mozambique, Mali) demonstrate the deployment paradox at full resolution. Zimbabwe shows -0.017 Delta-AUC (light green, near-zero performance change) yet achieves 29.1\% rescue rate (yellow, moderate) and +20.4pp recall gain (dark green, highest observed)---clear humanitarian justification despite aggregate metric degradation. DRC exhibits even starker contrast: -0.084 Delta-AUC (dark red, worst performance) alongside 48.2\% rescue rate (dark green, second-highest) and +14.2pp recall gain (green, substantial). Rescuing nearly half of AR's failures justifies the precision cost under 10:1 FN:FP humanitarian cost weighting that prioritises missed crises over false alarms.

Conversely, Minimal Benefit countries (Kenya, Ethiopia, Malawi) demonstrate why superficial Delta-AUC analysis misleads. Kenya achieves -0.002 Delta-AUC (green, minimal degradation)---seemingly favourable---but 3.3\% rescue rate (dark red, negligible) and +0.5pp recall gain (dark red, trivial) reveal cascade adds no humanitarian value. The near-zero Delta-AUC reflects AR baseline already performing well, not cascade success. This pattern holds for Ethiopia (4.0\% rescue, +0.9pp gain) and Malawi (4.8\% rescue, +2.9pp gain): low rescue rates disqualify deployment regardless of Delta-AUC sign.

Somalia emerges as a unique case: the lone positive Delta-AUC (+0.001, green) with modest rescue rate (18.2\%, yellow) but minimal recall gain (+1.4pp, orange). This rare configuration suggests cascade improves precision without substantially increasing false alarms, enabled by alignment between shock-driven crises (health/displacement) and sufficient news density (16.5\% health SHAP importance, highest elevation observed across all countries). Chad and Niger (0.0\% rescue rate, dark red) definitively demonstrate AR superiority: zero humanitarian benefit disqualifies cascade regardless of Delta-AUC values, confirming news deserts where coverage deficiency prevents any predictive gain.

\textbf{Evidence-based deployment criteria.} Integrating insights from scatter plot and benefit matrix yields operational classification rules that prioritise humanitarian impact over aggregate accuracy metrics: (1) \textit{Deploy cascade} in High Benefit countries (Zimbabwe, Sudan, DRC, Nigeria, Mozambique, Mali) where rescue rates exceed 15\% and recall gains exceed +3pp, accepting negative Delta-AUC as necessary cost of targeting rapid-onset shocks; (2) \textit{Consider conditional deployment} in moderate-benefit countries (Somalia 18.2\% rescue rate) where positive Delta-AUC with modest recall gains suggest potential value in specific crisis contexts; (3) \textit{Use AR baseline only} in Persistence-Dominated countries (Kenya, Ethiopia, Malawi) where rescue rates <5\% indicate that crises follow predictable temporal patterns well-captured by AR baselines, but news features would add value if coverage density increased; (4) \textit{Definitively avoid cascade} in News Desert countries (Chad, Niger with 0.0\% rescue rates) where insufficient media coverage prevents news features from providing humanitarian benefit due to data constraints rather than methodological limitations.

This evidence-based framework operationalizes the humanitarian principle that false negatives (missed crises) carry far greater cost than false positives (unnecessary alerts) in early-warning contexts. By prioritising lives saved (key saves, rescue rate, recall gain) over aggregate accuracy metrics (Delta-AUC, balanced accuracy), the classification aligns model deployment with operational priorities: preventing catastrophic failures (missed crises in Sudan/Zimbabwe/DRC) takes precedence over minimising average error (balanced accuracy optimisation) when the stakes involve famine, displacement, and mortality.

\section{Theoretical Implications}

\subsection{Rethinking News-Based Forecasting: The Autocorrelation Trap as Field-Wide Challenge}

Our findings challenge three foundational assumptions in news-based crisis prediction literature:

\textbf{Assumption 1: ``News provides substantial predictive value for crises.''} Challenged by: AR baseline achieving AUC=0.907 with zero news features, approaching published news-based models (achieving 93.8\% of Balashankar et al.'s PR-AUC). Most published work reports AUC 0.75-0.90 without AR comparisons, implicitly claiming news value while actually measuring autocorrelation.

\textbf{Revised understanding}: News provides marginal value (6.2\% at most) beyond persistence. The relevant question is not ``does news predict crises?'' but ``does news predict crises \textit{beyond what temporal and spatial autocorrelation already capture}?'' Most prior work cannot answer this question because AR baselines are absent.

\textbf{Assumption 2: ``Higher AUC = better early warning system.''} Challenged by: Our cascade framework achieves lower F1 (0.668 vs AR's 0.732) but rescues 249 critical cases that AR misses. Traditional metrics (AUC, F1) optimise average performance, but humanitarian early warning prioritises worst-case coverage (detecting the hardest-to-predict crises where lives are at stake).

\textbf{Revised understanding}: Evaluation metrics must align with operational priorities. For humanitarian contexts: Recall > Precision (missing crises is catastrophic), hard-case performance > average performance (rescuing AR failures matters more than incremental gains on easy cases), interpretability > black-box accuracy (understanding \textit{why} predictions change informs response strategies).

\textbf{Assumption 3: ``More features = better models.''} Challenged by: Standalone ablation shows XGBoost Advanced (35 features) achieves AUC=0.697 versus Ratio+Location (12 features, AUC=0.727). However, SHAP reveals z-score features account for 74.7\% of marginal attribution in combined models. Adding HMM (+0.007 AUC) and DMD (+0.002 AUC) reflects their design for specialized detection rather than universal discrimination improvement.

\textbf{Revised understanding}: Feature value depends on operational objectives. For \textit{maximizing discrimination on all AR-difficult cases}, the Ratio + Location ablation (9 ratio features + 3 location metadata) provides optimal complexity-performance trade-off. For \textit{interpretability and extreme event detection}, HMM transition risk captures regime shifts (\#5 feature ranking, 3.2\% importance) while DMD achieves the largest coefficient (+352.38) for complex emergencies. This demonstrates prediction-interpretability trade-offs: simpler models maximize discrimination metrics, while advanced models enable mechanistic understanding of crisis dynamics.

\subsection{Two-Component Crisis Dynamics: Low-Frequency Persistence vs High-Frequency Shocks}

Our results reveal that food security crises exhibit two-component dynamics, requiring distinct modelling strategies:

\textbf{Low-Frequency Component (Structural Persistence)}: The majority of IPC transitions (73.2\%, 3,895/5,322 crises) follow predictable temporal and spatial patterns:
    \begin{itemize}
    \item \textbf{Temporal persistence}: $\text{IPC}_{t} \approx \text{IPC}_{t-1}$ (crises persist across assessment periods).
    \item \textbf{Spatial clustering}: $\text{IPC}_i \approx \text{IPC}_j$ for neighboring districts $i, j$ (crises cluster geographically).
    \item \textbf{Slow deterioration}: Gradual transitions (IPC 2 $\times$ 2.5 $\times$ 3 over 6-12 months) that Lt and Ls capture effectively.
    \end{itemize}

This component represents \textbf{structural food insecurity}---chronic poverty, environmental degradation, weak infrastructure---that evolves slowly and predictably. AR baselines excel at predicting these cases because persistence assumptions hold.

\textbf{High-Frequency Component (Shock-Driven Transitions)}: The remaining 26.8\% (1,427/5,322 crises) exhibit rapid-onset dynamics:
    \begin{itemize}
    \item \textbf{Temporal breaks}: Sudden IPC jumps (1.5 $\times$ 3.5 within one period) that violate persistence assumptions.
    \item \textbf{Spatial isolation}: Localized shocks (district-specific conflicts, market collapses) that weak spatial signals.
    \item \textbf{Regime transitions}: Qualitative shifts (peaceful $\times$ violent, stable $\times$ crisis-prone) that historical averages miss.
    \end{itemize}

This component represents \textbf{shock-driven food insecurity}---conflicts, economic collapses, climate extremes---that unfold rapidly and unpredictably. AR baselines fail on these cases because simple extrapolation cannot anticipate structural breaks.

\textbf{Why Two-Stage Modelling Makes Theoretical Sense}: The autocorrelation trap arises when we apply a single model to both components:
    \begin{itemize}
    \item Training on all 20,722 observations: Low-frequency signal dominates (73.2\% of crises), overwhelming high-frequency signal (26.8\%). Models learn persistence, achieving high average accuracy but missing critical shocks.
    \item Result: AUC=0.907 for AR (excellent at persistence) vs AUC=0.697 for XGBoost on full data (trying to learn both components simultaneously, succeeding at neither).
    \end{itemize}

The two-stage framework separates components:
    \begin{itemize}
    \item \textbf{Stage 1 (AR baseline)}: Predicts low-frequency component (persistence). Achieves 0.907 AUC, correctly identifying 73.2\% of crises.
    \item \textbf{Stage 2 (News-based models)}: Targets high-frequency component (shocks). Trained on WITH\_AR\_FILTER (6,553 observations where IPC\textsubscript{t-1} $\leq$ 2 AND AR predicted non-crisis), learns shock signals without interference from persistence.
    \item \textbf{Cascade combination}: AR handles structural persistence; Stage 2 rescues shock-driven failures. Achieves 77.9\% recall (up from 73.2\%), prioritising detection of the hardest-to-predict, highest-stakes crises.
    \end{itemize}

This decomposition aligns with humanitarian operational needs: most resources deployed to high-confidence AR predictions (low-frequency crises with strong persistence signals), while specialised monitoring targets WITH\_AR\_FILTER cases (IPC\textsubscript{t-1} $\leq$ 2 AND AR=0) representing high-frequency shocks requiring dynamic surveillance.

\subsection{Geographic Heterogeneity: News Value is Context-Dependent}

The field's implicit assumption---that news features provide universal value across contexts---is empirically refuted by our findings. Three dimensions of heterogeneity:

\textbf{1. Coverage Heterogeneity}: News-based models can only predict what news covers. GDELT's English-language bias creates systematic coverage gaps:
    \begin{itemize}
    \item High-coverage countries (Sudan: 2,500+ articles/district/year): Rich signals, stable model performance (AUC 0.682).
    \item Low-coverage countries (Niger: <300 articles/district/year): Sparse signals, unstable performance (AUC 0.068).
    \end{itemize}

Implication: Deploying news-based models in low-coverage countries wastes computational resources without improving predictions. Coverage thresholds ($\geq$500 articles/district/year) should gate model selection.

\textbf{2. Crisis Type Heterogeneity}: Different crisis drivers produce different news signatures:
    \begin{itemize}
    \item \textbf{Conflict crises} (Sudan, DRC, Nigeria): Strong news signals (violence reporting, casualty counts, displacement). News features add value (59 + 40 + 27 = 126 key saves, 50.6\% of total).
    \item \textbf{Economic crises} (Zimbabwe): Moderate news signals (inflation reporting, policy analysis). News features add value (77 key saves, 30.9\%).
    \item \textbf{Climate crises} (Kenya pastoral, Madagascar): Weak news signals (regional droughts produce spatially correlated coverage that Ls already captures). News features add minimal value (8 + 0 = 8 key saves, 3.2\%).
    \end{itemize}

Implication: News-based models should be selectively deployed based on predominant crisis type. One-size-fits-all systems misallocate resources to contexts where news provides no marginal information.

\textbf{3. Temporal Heterogeneity}: Crisis dynamics vary not just across space but across time. Mixed-effects random slopes reveal that news feature contributions vary by country:
    \begin{itemize}
    \item Conflict-affected countries: conflict\_ratio and displacement\_ratio have large positive slopes (news coverage predicts IPC transitions).
    \item Structurally food-insecure countries: food\_security\_ratio has near-zero slope (reporting reactive, not predictive).
    \end{itemize}

This temporal heterogeneity suggests dynamic model weighting: adjust feature importance based on recent crisis history rather than using global fixed weights.

\section{Practical Implications for Food Security Early Warning Systems}

\subsection{Operational Deployment Considerations}

Our findings provide actionable guidance for humanitarian organisations deploying early warning systems:

\textbf{Decision Rule 1: Simple Binary Override Logic}

The cascade framework uses straightforward binary logic:

\textbf{Step 1}: Calculate AR baseline binary predictions for all 1,920 districts (requires only historical IPC data, computationally trivial). AR achieves 73.2\% recall with 73.2\% precision.

\textbf{Step 2}: For cases where AR predicts no crisis (AR = 0), deploy Stage 2 news-based analysis. This covers the WITH\_AR\_FILTER subset (IPC\textsubscript{t-1} $\leq$ 2 AND AR=0, comprising 6,553 cases where AR might miss emerging crises).

\textbf{Step 3}: Apply simple binary override rule:
\begin{itemize}
\item If AR = 1 (crisis predicted): \textbf{Trust AR, deploy resources immediately}
\item If AR = 0 (no crisis predicted):
    \begin{itemize}
    \item If Stage 2 = 1 (crisis detected): \textbf{Override to crisis (1)}, deploy resources
    \item If Stage 2 = 0 (no crisis detected): \textbf{Keep as no crisis (0)}, routine monitoring
    \end{itemize}
\end{itemize}

This binary logic (no probability thresholds, no complex cascading rules) maximises interpretability and operational simplicity. The override rate is 17.4\% of AR failures (249 crises rescued out of 1,427 AR-missed cases).

\textbf{Decision Rule 2: Country-Specific Deployment Thresholds}

Not all countries benefit equally from Stage 2 deployment. Set country-specific thresholds based on historical key save rates:

    \begin{itemize}
    \item \textbf{High-benefit countries} (Zimbabwe, Sudan, DRC): Deploy Stage 2 for all WITH\_AR\_FILTER cases (IPC\textsubscript{t-1} $\leq$ 2 AND AR=0). Historical rescue rate: 30.9\%, 23.7\%, 16.1\% respectively---high enough to justify full deployment.
    \item \textbf{Moderate-benefit countries} (Nigeria, Mali, Mozambique): Deploy Stage 2 for all WITH\_AR\_FILTER cases, but monitor cost-benefit ratio. Rescue rate: 10.8\%, 4.8\%, 6.0\%---lower but still operationally meaningful.
    \item \textbf{Low-benefit countries} (Niger, Kenya pastoral zones, Madagascar): Skip Stage 2 entirely, use AR baseline only. Rescue rate <3\%, insufficient to justify computational cost.
    \end{itemize}

This stratified deployment reduces false positives (by not applying Stage 2 where it adds noise) while preserving true positives (by deploying where it rescues crises).

\textbf{Decision Rule 3: Computational Efficiency Through Selective Application}

Rather than running Stage 2 universally, the framework applies it selectively only for AR=0 cases (cases where AR predicts no crisis). This reduces computational burden:

\begin{itemize}
\item \textbf{AR baseline}: Runs universally on all 20,722 observations (cheap, requires only IPC history)
\item \textbf{Stage 2 XGBoost}: Runs only on 6,553 WITH\_AR\_FILTER cases where AR=0 (31.6\% of data)
\item \textbf{Computational savings}: 68.4\% reduction in Stage 2 processing compared to universal deployment
\end{itemize}

This selective application concentrates computational resources where news features provide dominant marginal signal (the 26.8\% of shock-driven crises where AR fails), while avoiding unnecessary processing for persistence-dominated cases well-captured by AR baselines.

\subsection{When to Trust AR vs When to Apply Cascade Override}

The binary cascade logic automatically determines when to trust AR versus when to override:

\textbf{AR Baseline Trusted (No Override Possible)}:
    \begin{itemize}
    \item \textbf{When AR = 1 (crisis predicted)}: Framework always trusts AR's crisis predictions. These 5,322 cases (AR-detected crises) achieve 73.2\% precision, representing structurally persistent crises where temporal/spatial patterns provide strong signal. Deploy humanitarian resources immediately---no Stage 2 confirmation needed.
    \item \textbf{Geographic contexts where AR excels}: Persistence-dominated countries (Kenya pastoral zones, Ethiopia, Malawi) where climate-driven crises follow predictable seasonal patterns. Spatial autocorrelation (Ls) captures regional drought patterns effectively. In these contexts, Stage 2 provides less additional value.
    \end{itemize}

\textbf{Stage 2 Override Applied (AR=0 Cases)}:
    \begin{itemize}
    \item \textbf{When AR = 0 AND Stage 2 = 1}: The 249 key saves where AR missed a crisis but Stage 2 detected it through news signals. These represent shock-driven crises (conflict escalations, economic collapses, regime transitions) where temporal persistence breaks down and news features provide dominant predictive signal.
    \item \textbf{Geographic contexts where override succeeds}: Conflict-affected, news-dense countries (Zimbabwe 29.1\% rescue rate, Sudan 25.7\%, DRC 48.2\%, Mali 48.0\%) where rich media coverage enables shock detection. In these contexts, Stage 2 rescue rate justifies deployment.
    \end{itemize}

\textbf{Operational Protocol (Simple Binary Decision)}:
    \begin{itemize}
    \item \textbf{Red Alert}: AR = 1 OR Cascade = 1 (either system detects crisis) → Deploy humanitarian resources immediately (food aid, livelihood support, emergency funding mobilization)
    \item \textbf{Green Status}: AR = 0 AND Cascade = 0 (both agree: no crisis) → Routine monitoring, no immediate action required
    \end{itemize}

This simple two-tier system (crisis/no-crisis) maximises operational clarity and avoids ambiguous middle categories. The trade-off is precision decline (0.732 → 0.585) for recall improvement (0.732 → 0.779), which humanitarian cost-benefit analysis (10:1 FN:FP weighting) justifies: missing a crisis carries 10× worse consequences than a false alarm.

\subsection{Cost-Benefit of News Monitoring Infrastructure}

Implementing the cascade framework requires sustained investment in news data infrastructure:

\textbf{Direct Costs}:
    \begin{itemize}
    \item GDELT API access and storage (~\$2,000/year for 18-country coverage).
    \item Feature engineering pipeline (12-month rolling HMM/DMD computation, ~40 CPU-hours/month on AWS EC2 t3.xlarge, ~\$150/month).
    \item Model retraining (monthly XGBoost hyperparameter search with 5-fold CV, ~8 hours on GPU instance, ~\$50/month).
    \item \textbf{Total direct cost}: ~\$4,200/year.
    \end{itemize}

\textbf{Indirect Costs}:
    \begin{itemize}
    \item Technical staff time (data scientist to maintain pipeline, ~10\% FTE, ~\$15,000/year salary burden).
    \item Integration with existing systems (FEWSNET, WFP VAM, API development, one-time ~\$25,000).
    \item \textbf{Total indirect cost}: ~\$40,000 first year, ~\$15,000/year ongoing.
    \end{itemize}

\textbf{Benefits}: Quantifying humanitarian value of 249 key saves:
    \begin{itemize}
    \item Average district population affected per crisis: ~150,000 people (median from IPC population estimates).
    \item 249 key saves $\times$ 150,000 people = 37.35 million person-periods of crisis averted or mitigated.
    \item If 8-month early warning enables 20\% reduction in crisis severity (via preemptive food assistance, livelihood support, market stabilization), this translates to ~7.47 million person-periods of reduced suffering.
    \item Humanitarian cost savings: \$50/person for timely intervention vs \$200/person for emergency response (standard FEWSNET estimates). Savings: \$1.12 billion over 3-year study period.
    \end{itemize}

\textbf{Cost-Benefit Ratio}: \$1.12B savings / \$60K investment (3-year annualized) = 18,667:1 benefit-cost ratio. Even if our severity reduction estimate is 10$\times$ optimistic, cost-benefit remains highly favourable (1,867:1).

\textbf{Recommendation}: News monitoring infrastructure investment is justified for high-benefit countries (Sudan, Zimbabwe, DRC) where 70.7\% of key saves concentrate. For low-benefit countries (Niger, Madagascar, Kenya pastoral), due to the limited news coverage in some districts the model shows limited value---deploy resources to advanced NLP techniques instead: multilingual transformer models, local-language news integration, social media text mining, and automated event extraction to capture crisis signals in sparse-coverage contexts.

\subsection{Integration with Existing Humanitarian Systems}

Our framework complements, rather than replaces, existing early warning infrastructure:

\textbf{FEWSNET Integration}: FEWSNET currently relies on expert-driven Integrated Food Security Phase Classification (IPC) assessments combining multiple data sources and field reports. Our AR baseline + cascade framework provides:
    \begin{itemize}
    \item \textbf{Automated early warnings} 8 months ahead of IPC assessments (which typically occur every 4 months with 1-2 month publication lag). This extends warning horizon from current 3-4 months to 8+ months.
    \item \textbf{Geographic coverage expansion}: Automated system monitors all 1,920 districts continuously (covering all districts with sufficient IPC data), while FEWSNET expert assessments cover ~50-70 priority districts per country due to resource constraints.
    \item \textbf{Objective baselines}: AR predictions provide data-driven starting points for expert deliberations, reducing anchoring bias and ensuring systematic coverage.
    \end{itemize}

Integration pathway: Deploy AR baseline as FEWSNET's "Outlook Monitor" generating monthly district-level alerts. Experts review alerts, validate with field data, adjust using local knowledge. Cascade framework provides "second opinion" for ambiguous cases.

\textbf{WFP Vulnerability Analysis and Mapping (VAM)}: WFP's VAM system conducts household surveys to assess food security. Our framework provides:
    \begin{itemize}
    \item \textbf{Survey targeting}: Identify high-risk districts (AR >0.629 or Cascade rescues) for priority survey deployment.
    \item \textbf{Temporal triggering}: Trigger rapid assessments when HMM detects regime transitions or cascade overrides AR predictions, rather than relying solely on scheduled surveys.
    \item \textbf{Resource allocation optimisation}: Direct food assistance to districts with highest predicted crisis probability, maximising impact per dollar.
    \end{itemize}

Integration pathway: WFP's HungerMapLive platform already displays near-real-time hunger estimates. Incorporate AR baseline predictions as "8-Month Outlook" layer, cascade key saves as "Crisis Alert" notifications triggering field verifications.

\textbf{IPC Technical Working Groups (TWGs)}: National TWGs (comprising government, UN agencies, NGOs) produce official IPC classifications every 4 months. Our framework provides:
    \begin{itemize}
    \item \textbf{Evidence base}: Quantitative forecasts complementing qualitative expert assessments.
    \item \textbf{Disagreement flagging}: When our predictions diverge from expert consensus, triggers deeper investigation into causes (data quality issues vs genuine signals).
    \item \textbf{Accountability}: Retrospective validation compares predictions to actual IPC outcomes, enabling continuous improvement of both automated and expert systems.
    \end{itemize}

Integration pathway: Submit AR baseline and cascade predictions to TWGs 2 weeks before scheduled IPC assessments. TWGs consider forecasts alongside other data sources and field reports, using ensemble of all information sources for final classifications.

\section{Methodological Contributions}

\subsection{Two-Stage Residual Modelling Framework: A General Approach for Autocorrelated Outcomes}

Our framework's theoretical contribution extends beyond food security to any domain with temporally/spatially autocorrelated outcomes:

\textbf{The General Problem}: When outcomes $y_t$ exhibit strong autocorrelation ($\text{Cor}(y_t, y_{t-1}) > 0.8$), standard supervised learning approaches produce models that:
    \begin{itemize}
    \item Achieve high average accuracy by learning persistence ($\hat{y}_t \approx y_{t-1}$).
    \item Fail catastrophically on structural breaks (regime transitions, shocks, anomalies) where persistence assumptions fail.
    \item Obscure whether features $X$ provide value beyond autocorrelation---the autocorrelation trap.
    \end{itemize}

\textbf{The Two-Stage Solution}:
    \begin{enumerate}
    \item \textbf{Stage 1 (Baseline)}: Model persistence explicitly using autoregressive features (temporal autoregressive features $y_{t-1}, y_{t-2}, \ldots$, spatial autoregressive features for geo data, seasonal components). Evaluate baseline performance to establish which cases require complementary signals.

    \item \textbf{Stage 2 (Residual)}: Train specialised model on WITH\_AR\_FILTER subset (IPC$_{t-1} \leq 2$ AND AR=0) using features $X$. This subset represents cases where AR predicts no crisis, requiring shock detection capabilities.

    \item \textbf{Binary Cascade Logic}: Simple override rule: If AR = 1 (crisis predicted), keep prediction. If AR = 0 (no crisis predicted), use Stage 2's binary prediction to detect shock-driven crises AR missed.
    \end{enumerate}

\textbf{Advantages}:
    \begin{itemize}
    \item Separates low-frequency (persistence) from high-frequency (shocks) components, enabling specialised modelling.
    \item Quantifies marginal contribution of features $X$ beyond autocorrelation, avoiding autocorrelation trap.
    \item Improves hard-case performance (precision-recall on failures) while maintaining average accuracy (baseline handles majority of cases).
    \item Computationally efficient (expensive feature engineering applied selectively, not universally).
    \end{itemize}

\textbf{Applicability Beyond Food Security}:
    \begin{itemize}
    \item \textbf{Conflict forecasting}: Civil war recurrence is highly persistent (PITF data shows 60\% of conflicts persist from year $t$ to $t+1$). Two-stage approach: AR baseline predicts persistence, news/social media features predict escalations/de-escalations.
    \item \textbf{Epidemic surveillance}: Disease incidence autocorrelated due to contagion dynamics. AR baseline models disease spread curves; genomic/mobility data predicts regime shifts (new variants, superspreader events).
    \item \textbf{Financial forecasting}: Asset prices exhibit momentum (autocorrelation). AR baseline captures trends; news sentiment predicts structural breaks (market crashes, policy shocks).
    \item \textbf{Environmental monitoring}: Vegetation indices (NDVI) highly autocorrelated. AR baseline predicts seasonal cycles; advanced NLP can extract climate anomaly narratives (droughts, floods) from news text to complement temporal/spatial patterns.
    \end{itemize}

The framework is domain-agnostic---applicable whenever: (1) outcomes autocorrelated, (2) most cases follow persistence but minority exhibit shocks, (3) features $X$ hypothesized to predict shocks but contaminated by autocorrelation in full data.

\subsection{WITH\_AR\_FILTER Training Strategy: Selective Supervision}

Tra\-di\-tion\-al supervised learning uses all labelled data for training. Our WITH\_\allowbreak AR\_\allowbreak FILTER strategy se\-lec\-tive\-ly samples hard cases where baseline fails, pro\-duc\-ing spe\-cial\-ized models:

\textbf{The Strategy}:
    \begin{enumerate}
    \item Train AR baseline on full dataset (20,722 observations) $\times$ 0.907 AUC.
    \item Identify WITH\_AR\_FILTER subset: Cases where IPC$_{t-1} \leq 2$ AND AR = 0 (baseline predicts no crisis) $\times$ 6,553 observations (31.6\%).
    \item Train Stage 2 XGBoost \textit{only on these 6,553 cases}, focusing on shock detection where AR fails.
    \item Binary cascade: If AR = 1, keep prediction. If AR = 0, use Stage 2's binary prediction.
    \end{enumerate}

\textbf{Why This Works}:
    \begin{itemize}
    \item \textbf{Signal-to-noise ratio}: In full data, low-frequency signal (persistence, 73.2\% of cases) overwhelms high-frequency signal (shocks, 26.8\%). Models learn persistence, ignoring shocks. WITH\_AR\_FILTER removes easy persistent cases, amplifying shock signal.
    \item \textbf{Class imbalance correction}: Full data has 25.7\% crisis prevalence; WITH\_AR\_FILTER subset has 6.0\% crisis prevalence (more balanced after removing AR true positives). Reduces need for aggressive class weighting.
    \item \textbf{Feature relevance}: News features provide minimal value for persistent crises (where AR suffices) but substantial value for shocks. Training on shocks only maximises feature utilisation.
    \end{itemize}

\textbf{Comparison to Alternatives}:
    \begin{itemize}
    \item \textbf{Hard example mining} (computer vision): Identifies misclassified examples, reweights in next training iteration. Similar spirit but requires iterative retraining; WITH\_AR\_FILTER is one-shot.
    \item \textbf{Boosting} (AdaBoost, Gradient Boosting): Iteratively upweights misclassified cases. Improves hard-case performance but doesn't separate low/high frequency components.
    \item \textbf{Curriculum learning}: Trains on easy examples first, progresses to hard examples. Opposite of WITH\_AR\_FILTER (we skip easy, train only on hard).
    \end{itemize}

WITH\_AR\_FILTER is unique in completely partitioning data into persistence vs shock subsets, training separate specialised models.

\subsection{Stratified Spatial Cross-Validation: Rigorous Generalisation Testing}

Standard k-fold CV randomly partitions data, producing overoptimistic performance estimates for spatial data due to spatial autocorrelation leakage \citep{roberts2017cross}. Our stratified spatial CV prevents leakage:

\textbf{The Method}:
    \begin{enumerate}
    \item Cluster 1,920 districts into 5 geographically contiguous regions using k-means on (latitude, longitude).
    \item Each fold holds out one entire region (all districts within cluster, all time periods for those districts).
    \item Train on remaining 4 regions, test on held-out region.
    \item Performance estimates reflect true out-of-sample generalisation to unseen geographic areas.
    \end{enumerate}

\textbf{Why Spatial CV Matters}:
    \begin{itemize}
    \item \textbf{Random CV inflates performance}: If neighbouring districts split across train/test, spatial autocorrelation (Ls feature) leaks information. Model appears to generalise but actually exploits proximity.
    \item \textbf{Spatial CV deflates performance (correctly)}: Held-out regions have no nearby training districts. Models must generalise using global patterns (news features, country metadata), not local proximity.
    \item \textbf{Real-world relevance}: Operational deployment requires predicting crises in regions with sparse historical data (new conflict zones, data-poor countries). Spatial CV simulates this scenario.
    \end{itemize}

\textbf{Performance Impact}: Comparing random CV vs spatial CV on XGBoost Advanced:
    \begin{itemize}
    \item Random 5-fold CV: AUC=0.743$\pm$0.092 (optimistic due to leakage).
    \item Spatial 5-fold CV: AUC=0.697$\pm$0.175 (larger variance, lower mean, realistic).
    \end{itemize}

The 0.046 AUC gap represents leakage from spatial autocorrelation. Our spatial CV eliminates this, providing honest performance estimates. All results reported in this dissertation use spatial CV.

\subsection{Crisis-Focused HMM and DMD Feature Engineering}

Our feature engineering pipeline introduces two novel components for crisis detection:

\textbf{HMM for Regime Transition Detection}: Standard HMM applications (speech recognition, genomics) assume hidden states generate observations. We reverse this: use news features to infer \textit{crisis regime states}.

\textbf{Implementation}:
    \begin{itemize}
    \item States: 3 hidden states (stable, transitioning, crisis-prone), estimated via Baum-Welch algorithm \citep{rabiner1989tutorial}.
    \item Observations: 9-dimensional news vectors (conflict\_ratio, displacement\_ratio, ..., weather\_ratio) per district-month.
    \item Outputs: hmm\_ratio\_crisis\_prob (posterior probability of crisis-prone state), hmm\_ratio\_transition\_risk (probability of transitioning from stable to crisis within 3 months), hmm\_ratio\_entropy (state uncertainty).
    \end{itemize}

\textbf{Crisis-Specific Innovation}: Unlike standard HMM, we condition state transitions on crisis outcomes (IPC$\geq$3 vs IPC<3). This produces ``crisis-aware'' state definitions: stable = low IPC historically, crisis-prone = high IPC historically. Transition risk then measures probability of crossing IPC=3 threshold based on news narrative shifts.

\textbf{DMD for Temporal Mode Extraction}: DMD (from fluid dynamics \citep{schmid2010dynamic}) decomposes time series into exponential modes $\phi_k e^{\omega_k t}$ with growth rates $\omega_k$. We adapt for crisis prediction:

\textbf{Implementation}:
    \begin{itemize}
    \item Input: 12-month rolling window of 9 news ratios (12$\times$9=108-dimensional trajectory matrix $X$).
    \item DMD decomposition: $X \approx \sum_{k=1}^{9} \phi_k e^{\omega_k t}$ (9 modes extracted via SVD + eigenvalue decomposition).
    \item Outputs: dmd\_ratio\_crisis\_growth\_rate (largest positive $\omega_k$, indicating fastest-growing narrative), dmd\_ratio\_crisis\_instability (variance of $\omega_k$, measuring temporal volatility), dmd\_ratio\_crisis\_frequency (imaginary part of $\omega_k$, oscillation frequency), dmd\_ratio\_crisis\_amplitude ($\|\phi_k\|$, mode magnitude).
    \end{itemize}

\textbf{Crisis-Specific Innovation}: We select ``crisis mode'' as the mode $\phi_k$ with highest correlation to IPC$\geq$3 outcomes in training data. This focuses DMD on crisis-relevant temporal patterns (conflict escalations, displacement surges) rather than all variations in news coverage.

\textbf{Data requirements}: HMM requires 6+ months for state convergence (Baum-Welch iterative); DMD requires 12-month windows for robust mode estimation. Both methods achieve high convergence rates (HMM: 89.5\%, DMD: 88.7\%) but exclude observations with sparse news coverage (<200 articles/year), producing 10.6\% observation loss. For operational deployment: HMM transition risk (3.2\% importance, \#5 feature ranking) captures interpretable regime shifts applicable to most districts; DMD achieves largest coefficient (+352.38) but targets rare extreme events (<3\% observations), making it valuable for catastrophic crisis detection but limited for universal deployment.

\section{Limitations}

\subsection{Data Coverage Heterogeneity and Systematic Bias}

Our analysis spans 18 countries and 1,920 districts (from 3,438 districts in the raw IPC database), but coverage is uneven:

\textbf{Geographic Gaps}:
    \begin{itemize}
    \item 10 countries excluded (Cameroon, Burkina Faso, Burundi, Chad, Central African Republic, Angola, Mauritania, Lesotho, Rwanda, Togo) due to insufficient GDELT coverage (<200 articles/district/year threshold).
    \item Within included countries, urban districts over-represented (capital cities average 5,000+ articles/year) while rural pastoral zones under-represented (Turkana County, Kenya: 180 articles/year, below threshold).
    \item Conflict zones have paradoxical coverage: active war zones (South Sudan, DRC Ituri) may have \textit{lower} coverage than moderately unstable regions (Sudan) due to journalist safety concerns.
    \end{itemize}

\textbf{Temporal Gaps}:
    \begin{itemize}
    \item IPC assessments occur every 4 months, producing 20,722 district-period observations across 1,920 districts over 48 months (2021-2024). Rapid-onset crises may emerge and resolve between assessments, missing our ground truth labels.
    \item GDELT coverage quality varies: 2021 data richer than 2020 (COVID reporting surge increased African coverage).
    \end{itemize}

\textbf{Systematic Bias Implications}:
    \begin{itemize}
    \item News-dense countries (Sudan, Zimbabwe, Kenya) over-represented in key saves analysis. Our claim that "Sudan benefits most from news features" may reflect data availability, not genuine crisis dynamics.
    \item Rural crises under-detected: Pastoral mobility, remote agricultural failures, and localized conflicts in data-poor regions may be systematically missed.
    \item External validity limited: Generalisation to excluded countries uncertain. Burkina Faso, Chad, and Central African Republic (all excluded) face severe crises but lack sufficient news coverage for our methods.
    \end{itemize}

\subsection{English-Language News Bias and GDELT Limitations}

GDELT monitors English-language news sources, introducing linguistic and cultural biases:

\textbf{Language Bias}:
    \begin{itemize}
    \item French-speaking countries (DRC, Niger, Mali, Burkina Faso) rely on local French-language media. GDELT's English-only coverage captures international reporting (Reuters, AFP in English) but misses domestic discourse (local radio, regional newspapers).
    \item Arabic-speaking regions (Sudan, Somalia) similarly under-represented. International coverage focuses on major events (Khartoum conflicts) while missing localized crises in Darfur, Kordofan.
    \item Amharic (Ethiopia), Swahili (Kenya, Tanzania), Portuguese (Mozambique, Angola) media ecosystems largely invisible to GDELT.
    \end{itemize}

\textbf{Editorial Bias}:
    \begin{itemize}
    \item Western media over-represent crises with Western aid involvement (Somalia famine 2011, South Sudan displacement) while under-representing crises without international attention (Madagascar chronic malnutrition, Malawi food insecurity).
    \item Conflict-driven crises receive disproportionate coverage (Nigeria Boko Haram, Sudan Darfur) versus silent emergencies (Zimbabwe economic collapse, Madagascar cyclones).
    \end{itemize}

\textbf{Implications for Findings}:
    \begin{itemize}
    \item Key saves concentration in Sudan (59 saves), DRC (40 saves) may reflect GDELT's strength in covering conflict zones with English-language international reporting, not genuine superiority of news features in these contexts.
    \item Madagascar (0 key saves) and Malawi (3 saves) may suffer from coverage bias, not genuine absence of news value---if we had Malagasy or Chichewa media, performance might improve.
    \end{itemize}

\textbf{Mitigation Strategies} (not implemented in this study, future work):
    \begin{itemize}
    \item Integrate multi\-lin\-gual news sources (Factiva, Lex\-is\-Nex\-is with Arabic/\allowbreak French/\allowbreak Portuguese coverage).
    \item Partner with local media monitoring organisations (e.g., African Media Barometer, local radio transcription services).
    \item Use machine translation (Google Translate API) to incorporate non-English GDELT coverage (currently excluded).
    \end{itemize}

\subsection{IPC Assessment Delays and Temporal Resolution Constraints}

Our ground truth (IPC classifications) has inherent limitations:

\textbf{Retrospective Nature}:
    \begin{itemize}
    \item IPC assessments published 1-3 months after reference period ends (e.g., October 2022 IPC published December 2022-January 2023). By the time "early warnings" would be issued (8 months before IPC period), the outcome is not yet observed.
    \item Our retrospective analysis uses hindcasting with spatial cross-validation across the full 2021-2024 temporal span. Operational deployment requires true forecasting (predict December 2024 IPC using April 2024 data), which we cannot validate until IPC published mid-2025.
    \end{itemize}

\textbf{Temporal Resolution}:
    \begin{itemize}
    \item IPC periods last 4 months (e.g., October 2022-January 2023 is single observation). Crises emerging and resolving within one period (e.g., 2-month displacement crisis December 2022-January 2023) are masked by period-level aggregation.
    \item Our h=8 (32 weeks) horizon predicts period-level IPC, not month-level dynamics. Finer temporal resolution (monthly IPC estimates from FEWSNET Food Security Outlook) would enable monthly predictions but introduces label noise (FEWSNET outlooks are projections, not observations).
    \end{itemize}

\textbf{Assessment Quality Variation}:
    \begin{itemize}
    \item IPC Technical Working Groups vary in capacity and data access. South Sudan TWG (well-funded, UN-supported) produces high-quality assessments; Madagascar TWG (under-resourced) may have classification errors.
    \item During COVID-19 (2020-2021), field assessments reduced, relying more on remote sensing and key informants. This may introduce systematic errors in 2021 IPC labels (our training data).
    \end{itemize}

\textbf{Implications}:
    \begin{itemize}
    \item True model performance may differ from reported results if IPC labels contain errors (e.g., undetected crises classified as IPC 2 when actually IPC 3). Our models predict noisy ground truth, not true latent food security.
    \item Temporal resolution limits operational utility: 8-month predictions at 4-month IPC period granularity provide coarse warnings. Humanitarian actors need monthly or even weekly forecasts for resource allocation.
    \end{itemize}

\subsection{8-Month Horizon Constraints and Horizon-Dependent Dynamics}

We focus on h=8 (32 weeks, ~8 months) forecast horizon based on FEWSNET operational needs. However:

\textbf{Horizon-Dependent Performance}:
    \begin{itemize}
    \item AR baseline: h=4 (AUC 0.921, Precision 0.762), h=8 (AUC 0.907, Precision 0.732), h=12 (AUC 0.889, Precision 0.687). Performance degrades with longer horizons (expected: prediction harder farther into future).
    \item News features may have \textit{different} horizon-dependent dynamics: short-horizon (h=4) predictions may benefit from immediate news spikes, while long-horizon (h=12) predictions require sustained narrative shifts that HMM captures better.
    \end{itemize}

\textbf{We Do Not Optimise Across Horizons}:
    \begin{itemize}
    \item All ablation studies, cascade tuning, and interpretability analysis conducted at h=8 only. Optimal feature set may differ for h=4 (favour z-scores for short-term spikes?) or h=12 (favour HMM for long-term transitions?).
    \item Multi-horizon optimisation (jointly tuning models for h=4, 8, 12 to maximise average performance) left for future work.
    \end{itemize}

\textbf{Operational Mismatch}:
    \begin{itemize}
    \item Humanitarian response timelines vary: emergency food aid (4-week mobilisation), livelihood programs (12-week planning), development interventions (24-week initiation). Single h=8 forecast may not align with all response modalities.
    \item Ideally: provide horizon-specific predictions (h=4 for emergency response, h=8 for preparedness, h=12 for development planning). Our framework supports this (can train separate models per horizon) but we only implement h=8.
    \end{itemize}

\subsection{Precision Trade-Off and Operational Alert Fatigue}

The cascade framework's precision reduction (from 0.732 to 0.585, a 14.7-percentage-point drop) has operational consequences:

\textbf{Alert Fatigue Risk}:
    \begin{itemize}
    \item 41.5\% of cascade crisis predictions are false positives (2,939 FP out of 7,083 positive predictions). If humanitarian actors deploy resources to all cascade alerts, 41.5\% of deployments are "wasted" (crisis does not materialize).
    \item Repeated false alarms erode trust in EWS. If field staff consistently find that cascade alerts do not correspond to actual crises, they may ignore future warnings---the "crying wolf" problem.
    \end{itemize}

\textbf{Resource Allocation Challenges}:
    \begin{itemize}
    \item Preemptive food aid costs ~\$50/person (FEWSNET estimates). 2,939 false positives $\times$ 150,000 average district population $\times$ \$50 = \$22 billion hypothetical cost if all alerts trigger full deployment.
    \item In practice, organisations use tiered responses (monitoring $\times$ standby $\times$ deployment), mitigating costs. But even standby operations (enhanced surveillance, staff travel, partner coordination) impose non-trivial costs.
    \end{itemize}

\textbf{Humanitarian vs ML Evaluation Divergence}:
    \begin{itemize}
    \item ML metrics (F1=0.668 for cascade vs 0.732 for AR) suggest cascade is worse. But humanitarian cost-sensitive evaluation (10:1 FN:FP weighting) favours cascade.
    \item This tension reflects deeper question: \textit{who decides cost ratios?} We assume 10:1 based on FEWSNET guidance, but individual organisations may have different tolerances (budget-constrained NGOs may prefer 5:1, well-funded UN agencies may accept 20:1).
    \end{itemize}

\textbf{Mitigation Strategies}:
    \begin{itemize}
    \item Future extension: Implement tiered alerts (red/orange/yellow based on model confidence scores) to differentiate high-confidence from low-confidence predictions, allowing resource allocation proportional to risk. Current binary system (Red Alert vs Green Status) prioritises simplicity.
    \item Retrospective performance reporting: publish monthly "cascade performance dashboards" showing recent precision/recall, enabling organisations to calibrate their response thresholds based on observed accuracy.
    \item Ensemble with other EWS: combine cascade predictions with FEWSNET expert outlooks and WFP HungerMapLive. Deploy only when multiple systems agree, reducing false positive rate.
    \end{itemize}

\subsection{External Validity: Africa-Specific Findings, Uncertain Generalisation}

All 20,722 observations come from 18 African countries. Generalisation to other regions uncertain:

\textbf{Africa-Specific Crisis Dynamics}:
    \begin{itemize}
    \item Conflict patterns: African conflicts often linked to resource competition (pastoral land, mining), ethnic politics, and weak state capacity. Asia/Latin America conflicts may have different drivers (ideology, drug trade, border disputes) producing different news signatures.
    \item Climate vulnerability: Africa disproportionately affected by droughts, with limited irrigation infrastructure. South Asia (monsoon-dependent) or Caribbean (hurricane-prone) have different climate-food security dynamics.
    \item News ecosystems: GDELT coverage density higher in Anglophone Africa (Kenya, Nigeria, Zimbabwe) due to colonial legacy. Middle East, Central Asia, Latin America have different media landscapes.
    \end{itemize}

\textbf{IPC vs Other Food Security Metrics}:
    \begin{itemize}
    \item IPC specific to humanitarian contexts (conflict zones, fragile states). Developed countries use different metrics (USDA Food Security Scale, FAO Food Insecurity Experience Scale). Our methods may not transfer to predicting these alternative outcomes.
    \item IPC emphasizes acute food insecurity (sudden crises). Chronic malnutrition (stunting, wasting) may have different predictive signals (long-term economic development, health infrastructure) that news features miss.
    \end{itemize}

\textbf{Implications}:
    \begin{itemize}
    \item Deploying our framework in Yemen, Syria, Afghanistan (non-African conflict zones) requires retraining on local data---cannot assume model weights transfer.
    \item Applying to chronic food insecurity prediction (e.g., Haiti, Guatemala, Bangladesh) may require different feature engineering (economic indicators, health metrics) beyond news.
    \item Cross-regional validation (train on Africa, test on Asia) would quantify generalisation, but lack of comparable IPC data outside Africa prevents this analysis.
    \end{itemize}

\section{Comparison to Related Work}

\subsection{This Work vs Balashankar et al. (2023): Methodological Divergences}

\citet{balashankar2023toward} represent the closest precedent for news-based food security prediction. Key differences:

\textbf{1. AR Baseline Comparison}:
    \begin{itemize}
    \item \textbf{Balashankar et al.}: Report PR-AUC=0.82 for news-based Random Forest models without AR baseline comparison. Implicitly claims news provides substantial value.
    \item \textbf{This work}: AR baseline achieves AUC=0.907, \textit{approaching} published news-based models (93.8\% of Balashankar's PR-AUC). Demonstrates that most of Balashankar's reported performance may reflect autocorrelation, not news features.
    \end{itemize}

\textbf{2. Training Strategy}:
    \begin{itemize}
    \item \textbf{Balashankar et al.}: Train Random Forest on all observations (full IPC time series), learning temporal patterns from autocorrelated sequences.
    \item \textbf{This work}: WITH\_AR\_FILTER selectively trains on AR failures only (6,553 / 20,722 observations), isolating high-frequency shock signal from low-frequency persistence.
    \end{itemize}

\textbf{3. Feature Engineering}:
    \begin{itemize}
    \item \textbf{Balashankar et al.}: Frame-semantic parsing for semantic content extraction + word embeddings for similarity. No explicit crisis-focused transformations.
    \item \textbf{This work}: Ratio features (compositional shifts), z-score features (anomalies), HMM (regime transitions), DMD (temporal modes). All engineered specifically for crisis prediction, not generic NLP.
    \end{itemize}

\textbf{4. Evaluation Rigor}:
    \begin{itemize}
    \item \textbf{Balashankar et al.}: Random train-test split (80/20), likely suffers from spatial autocorrelation leakage. Performance may be overestimated.
    \item \textbf{This work}: Stratified spatial 5-fold CV, geographic holdout prevents leakage. Lower reported performance (93.8\% of Balashankar’s PR-AUC) but more honest.
    \end{itemize}

\textbf{5. Interpretability}:
    \begin{itemize}
    \item \textbf{Balashankar et al.}: Limited feature importance analysis beyond model comparison. Cannot answer "when do news features matter?"
    \item \textbf{This work}: Three-method interpretability (XGBoost gain-based importance, mixed-effects coefficients, SHAP values), geographic heterogeneity analysis, case studies. Explicitly answers where/when news helps. \textbf{Critical revelation}: SHAP fundamentally reorders feature rankings (z-scores 74.7\% attribution vs location 2.6\%, despite location's 40.4\% tree-based importance), demonstrating that split frequency $\neq$ predictive contribution.
    \end{itemize}

\textbf{Conclusion}: Balashankar et al.'s claims about news value require reassessment in light of AR baseline comparisons. Their Random Forest model likely learned persistence (captured by Lt/Ls), not news signals. Our two-stage framework provides methodologically rigorous alternative, separating autocorrelation from genuine news contribution.

\subsection{This Work vs Traditional Early Warning Systems (FEWSNET, WFP)}

Existing operational EWS rely on expert-driven qualitative assessments:

\textbf{FEWSNET Approach}:
    \begin{itemize}
    \item Monthly Food Security Outlook reports synthesise diverse data sources, field reports, and expert judgment.
    \item Strengths: Incorporates local knowledge, flexible interpretation, trusted by donors/governments.
    \item Limitations: Labour-intensive (requires country analysts, field missions), subjective (inter-analyst agreement varies), limited geographic coverage (priority districts only), publication delays (1-2 months after reference period).
    \end{itemize}

\textbf{WFP HungerMapLive}:
    \begin{itemize}
    \item Near-real-time hunger estimates using household surveys and diverse real-time data sources.
    \item Strengths: High temporal resolution (daily updates), wide geographic coverage (120+ countries), objective metrics.
    \item Limitations: Retrospective (measures current hunger, not future crises), relies on self-reported consumption (social desirability bias), requires mobile network infrastructure (excludes remote areas).
    \end{itemize}

\textbf{Our Contribution Relative to Operational Systems}:

    \begin{table}[htbp]
    \centering
\caption{Comparison of Early Warning Approaches}
\resizebox{\textwidth}{!}{%
    \begin{tabular}{lccc}
\toprule
\textbf{Characteristic} & \textbf{FEWSNET} & \textbf{WFP HungerMapLive} & \textbf{This Work (AR + Cascade)} \\
\midrule
Forecast Horizon & 3-4 months & Real-time (0 months) & 8 months \\
Geographic Coverage & Priority districts & 120 countries & 18 countries (Africa) \\
Temporal Resolution & Monthly & Daily & 4-6 month IPC periods \\
Automation & Expert-driven & Semi-automated & Fully automated \\
Interpretability & Narrative reports & Dashboard metrics & Feature importance + SHAP \\
Data Sources & Multi-source & Surveys + mobile & News (GDELT only) \\
Validation & Retrospective (IPC) & Concurrent (surveys) & Retrospective (IPC) \\
\bottomrule
    \end{tabular}%
}
    \end{table}

\textbf{Complementarity, Not Replacement}:
    \begin{itemize}
    \item Our framework extends forecast horizon (8 months vs FEWSNET's 3-4), enabling earlier interventions.
    \item FEWSNET/WFP provide ground truth validation (expert assessments confirm/refute automated predictions).
    \item Integration strategy: Use our framework as "Outlook Monitor" generating alerts $\times$ FEWSNET experts investigate flagged districts $\times$ WFP deploys rapid assessments $\times$ Combined intelligence informs response decisions.
    \end{itemize}

\subsection{Positioning in ML for Social Good Literature}

Our work contributes to growing ML for Social Good (ML4SG) literature applying machine learning to humanitarian challenges:

\textbf{Crisis Informatics} \citep{imran2015processing}: Using social media (Twitter, Facebook) for disaster response (earthquake damage assessment, flood mapping). Our news-based approach shares data philosophy (leverage digital traces) but targets prediction (8 months ahead) vs response (real-time).

\textbf{Conflict Forecasting} \citep{mueller2021quantifying, oswald2020predicting}: Predicting civil conflicts using news and diverse indicators. Methodological parallel: autocorrelation trap affects conflict prediction (wars persist) just as food security (crises persist). Our two-stage framework directly transferable to conflict domain using NLP-extracted conflict event features.

\textbf{Poverty Mapping} \citep{jean2016combining}: Combining diverse data sources with household surveys to estimate poverty at fine spatial scales. Complements our work: poverty is slow-changing (low-frequency), food security is shock-driven (high-frequency). Together enable comprehensive vulnerability assessment.

\textbf{Climate-Informed Food Security} \citep{hsiang2013quantifying}: Linking temperature/precipitation to agricultural yields and food security. Our news features capture human responses (conflict, displacement, policy) that climate models miss. Combined climate + news models promising future direction.

\textbf{Unique Contribution}: Most ML4SG work reports absolute performance without autocorrelation-aware baselines. Our methodological critique (autocorrelation trap, two-stage framework, WITH\_AR\_FILTER) applicable across ML4SG domains:
    \begin{itemize}
    \item Conflict prediction: AR baseline = "conflict continues if ongoing, remains peaceful if peaceful." News features must beat this.
    \item Epidemic forecasting: AR baseline = SIR/SEIR models (disease dynamics). Genomic/mobility data must beat mechanistic models.
    \item Poverty prediction: AR baseline = "poverty tomorrow = poverty today" (very strong due to structural persistence). Satellite/social media must demonstrate marginal value.
    \end{itemize}

Our work provides template: (1) establish rigorous baseline capturing autocorrelation, (2) quantify marginal contribution of proposed features, (3) deploy two-stage framework prioritising hard cases. This template raises methodological bar for ML4SG claims.

\section{Future Research Directions}

\subsection{Real-Time Deployment and Operational Monitoring}

Our analysis is retrospective using spatial cross-validation across 2021-2024 data. Operational deployment requires real-time forecasting:

\textbf{Technical Requirements}:
    \begin{itemize}
    \item \textbf{GDELT API integration}: Automated daily ingestion of GDELT Event Database and Global Knowledge Graph. Current analysis uses static CSV exports; real-time requires streaming infrastructure.
    \item \textbf{Feature pipeline automation}: HMM and DMD require 12-month rolling windows, must be recomputed monthly as new data arrives. Current pipeline is batch (one-time computation); needs refactoring for incremental updates.
    \item \textbf{Model retraining cadence}: XGBoost hyperparameters tuned via cross-validation on the full dataset. How often to retrain for operational deployment? Monthly (captures evolving patterns but risks overfitting)? Annually (stable but may miss regime shifts)? Optimal cadence unknown.
    \item \textbf{IPC ground truth delays}: IPC assessments published 1-3 months after reference period ends. Real-time validation requires waiting 9-11 months (8-month forecast + 1-3 month publication lag) to confirm accuracy. How to maintain system trust during validation lag?
    \end{itemize}

\textbf{Research Questions}:
    \begin{enumerate}
    \item \textbf{Concept drift detection}: When do models become stale? Monitor prediction calibration (Brier score, log loss) on recent IPC outcomes. If calibration degrades >10\%, trigger retraining.
    \item \textbf{Automated performance reporting}: Generate monthly dashboards showing: cascade precision/recall trends, key save counts by country, false positive rates. Enables operational learning.
    \item \textbf{Human-in-the-loop integration}: When cascade overrides AR, flag for expert review before issuing public alert. Experts validate using field intelligence; feedback loop improves future model weights.
    \end{enumerate}

\textbf{Pilot Deployment Pathway}:
    \begin{itemize}
    \item \textbf{Phase 1 (Shadow mode, 6 months)}: Deploy system internally within FEWSNET/WFP, generating predictions but not issuing public alerts. Compare predictions to expert forecasts, measure agreement rates.
    \item \textbf{Phase 2 (Limited release, 12 months)}: Issue predictions for 3 pilot countries (Sudan, Zimbabwe, Kenya---high key save rates). Coordinate with national IPC TWGs for validation.
    \item \textbf{Phase 3 (Full deployment, ongoing)}: Expand to all 18 countries, integrate into FEWSNET Outlook Monitor and WFP early warning dashboards.
    \end{itemize}

\subsection{Advanced NLP Enhancement: Beyond Current Approach}

Current approach of news features rescue 17.4\% of AR failures, leaving 82.6\% undetected. Advanced NLP techniques offer substantial enhancement opportunities:

\textbf{Transformer-Based Semantic Understanding}:
    \begin{itemize}
    \item \textbf{BERT fine-tuning}: Fine-tune pre-trained BERT/RoBERTa on crisis-specific corpora (FEWSNET reports, humanitarian situation reports, IPC assessments) to capture domain-specific semantic patterns that generic bag-of-words features miss.
    \item \textbf{Contextual embeddings}: Replace simple article counts with contextualized text representations capturing nuanced crisis narratives (e.g., distinguishing "food aid arrived" from "food aid blocked").
    \item \textbf{Crisis-specific pre-training}: Build domain-adapted language model using large-scale news corpora + humanitarian reports as training corpus, enabling better understanding of crisis discourse.
    \end{itemize}

Integration strategy: Generate BERT embeddings for each district-month's news corpus (pooled [CLS] token), add as features to XGBoost. Hypothesis: semantic understanding rescues narrative-driven crises (policy changes, conflict escalations) that word counts miss.

\textbf{Multilingual NLP for Regional Coverage}:
    \begin{itemize}
    \item \textbf{mBERT/XLM-RoBERTa}: Multilingual transformers capturing French (Sahel, DRC, Madagascar), Arabic (Sudan, Somalia), Swahili (Kenya, Tanzania) news currently excluded from English-only GDELT.
    \item \textbf{Cross-lingual transfer}: Fine-tune on high-resource English crisis data, transfer to low-resource French/Arabic/Swahili via zero-shot learning.
    \item \textbf{Coverage expansion}: Integrate African local news sources (AllAfrica.com, regional newspapers) providing richer coverage than international English media.
    \end{itemize}

Integration strategy: Apply mBERT to multilingual news streams, concatenate language-specific embeddings with English features. Hypothesis: multilingual coverage rescues low-coverage contexts (Niger, Mali) where English-only GDELT is sparse.

\textbf{Social Media Text Mining for Real-Time Signals}:
    \begin{itemize}
    \item \textbf{Twitter crisis detection}: Fine-tune DistilBERT on disaster-specific Twitter datasets (CrisisNLP, HumAID) to extract real-time crisis signals from social media discussions.
    \item \textbf{Facebook community monitoring}: Analyse humanitarian organisation Facebook pages (WFP, UNICEF country offices) for early crisis mentions.
    \item \textbf{Temporal advantage}: Social media provides higher temporal resolution (hourly updates) than traditional news (daily), enabling faster crisis signal detection.
    \end{itemize}

Integration strategy: Add social\_media\_crisis\_score feature based on BERT-classified crisis-related social media posts. Hypothesis: social media captures rapid-onset crises (conflict escalations, sudden market disruptions) faster than traditional news.

\textbf{Automated Event Extraction and Knowledge Graphs}:
    \begin{itemize}
    \item \textbf{Named Entity Recognition (NER)}: Extract structured crisis events (WHO attacked WHOM in WHERE, WHAT food shortage in WHICH district) using transformer-based NER models (SpaCy, Stanza).
    \item \textbf{Relation extraction}: Identify causal relationships ("drought caused crop failure", "conflict displaced population") using dependency parsing and relation classification.
    \item \textbf{Knowledge graph construction}: Build temporal knowledge graphs linking entities (districts, armed groups, food commodities) and events, enabling graph neural network approaches.
    \end{itemize}

Integration strategy: Extract event features (attack\_frequency, displacement\_mentions, food\_shortage\_severity) from structured event extraction. Hypothesis: event-based features provide more precise crisis signals than aggregate article counts, rescuing crises with specific trigger events.

\textbf{NLP Fusion Architecture}:
    \begin{itemize}
    \item \textbf{Feature-level fusion}: Concatenate bag-of-words, BERT embeddings, multilingual features, social media scores, and event extraction outputs into unified feature vector for XGBoost.
    \item \textbf{Model-level fusion}: Train separate models for each NLP approach, ensemble via stacking with meta-learner optimising combination weights.
    \item \textbf{Attention-based fusion}: Use transformer architecture with cross-attention between different text representations, learning adaptive weights per representation per prediction.
    \end{itemize}

Research question: Which NLP enhancement provides highest marginal rescue rate for remaining 82.6\% of AR failures? Hypothesis: multilingual coverage (Niger, Mali) and event extraction (conflict-driven crises) offer largest gains.

\subsection{Multi-Horizon Optimisation: Joint Forecasting Across h=4, 8, 12}

Current framework optimises for h=8 only. Humanitarian response requires multiple horizons:

\textbf{Horizon-Specific Use Cases}:
    \begin{itemize}
    \item \textbf{h=4 (16 weeks, 4 months)}: Emergency response planning (food aid procurement, logistics mobilisation). Shorter horizon allows less lead time but higher accuracy (AR baseline AUC 0.921).
    \item \textbf{h=8 (32 weeks, 8 months)}: Preparedness and mitigation (pre-positioning supplies, livelihood programs, market support). Our current focus.
    \item \textbf{h=12 (48 weeks, 12 months)}: Development and resilience interventions (agricultural inputs distribution, infrastructure investments, social protection scaling). Longest lead time, enables preventive action but lower accuracy (AR baseline AUC 0.889).
    \end{itemize}

\textbf{Multi-Horizon Modelling Approaches}:
    \begin{enumerate}
    \item \textbf{Independent models}: Train separate XGBoost per horizon (h=4, h=8, h=12). Simple but ignores cross-horizon dependencies (a h=4 crisis forecast should inform h=8 forecast for same district).

    \item \textbf{Multi-task learning}: Single neural network with shared hidden layers, separate output heads per horizon. Shared representations capture common patterns (conflict signals relevant for all horizons), task-specific heads capture horizon-specific dynamics.

    \item \textbf{Sequential refinement}: Train h=12 model first (coarse long-term forecast), use h=12 predictions as features for h=8 model, use h=8 predictions as features for h=4 model. Exploits temporal dependency (long-term trends constrain short-term dynamics).
    \end{enumerate}

\textbf{Joint Optimisation Objective}:
    \begin{itemize}
    \item Maximise weighted average AUC: $0.3 \times \text{AUC}_{h=4} + 0.5 \times \text{AUC}_{h=8} + 0.2 \times \text{AUC}_{h=12}$ (weights reflect operational priority: h=8 most important).
    \item OR: Maximise minimum AUC: $\min(\text{AUC}_{h=4}, \text{AUC}_{h=8}, \text{AUC}_{h=12})$ (ensures robustness across all horizons, no single horizon catastrophically fails).
    \end{itemize}

\textbf{Research Questions}:
    \begin{enumerate}
    \item Do different news features matter at different horizons? (Hypothesis: z-scores matter for h=4 short-term spikes, HMM transitions matter for h=12 long-term regime shifts).
    \item Can multi-task learning improve h=8 performance by leveraging h=4 and h=12 auxiliary tasks?
    \item What is optimal temporal resolution for operational deployment? (Monthly predictions? Quarterly? Per IPC assessment period?)
    \end{enumerate}

\subsection{Causal Inference and Counterfactual Analysis: Beyond Prediction}

Our work is purely predictive: given news features $X$ at time $t$, predict IPC at $t+8$ months. Causal questions remain unanswered:

\textbf{Causal Questions}:
    \begin{enumerate}
    \item \textbf{Does conflict news coverage \textit{cause} food insecurity}, or merely reflect it? (Reverse causality: crises generate news, not news predicting crises).

    \item \textbf{Would intervening to reduce conflict} (peacekeeping, mediation) prevent predicted food crises? (Treatment effect estimation).

    \item \textbf{What is the marginal contribution of news-triggered early warnings} to humanitarian outcomes? (Counterfactual: what would have happened without our cascade framework alerts?).
    \end{enumerate}

\textbf{Methodological Approaches}:
    \begin{itemize}
    \item \textbf{Granger causality tests}: Does conflict\_ratio at $t$ improve prediction of IPC at $t+k$ beyond IPC history alone? Tests predictive causality (not true causality, but stronger than correlation).

    \item \textbf{Instrumental variables}: Use exogenous conflict shocks (political assassinations, border incidents, election violence) as instruments for conflict\_ratio. Estimate causal effect of conflict reporting on IPC transitions.

    \item \textbf{Difference-in-differences}: Compare IPC outcomes in districts receiving early warnings (treatment) vs matched control districts. Requires operational deployment data (which districts received FEWSNET alerts based on our predictions?).

    \item \textbf{Regression discontinuity}: Exploit classification threshold (0.629 probability cutoff for converting AR/Stage 2 probabilities to binary predictions). Districts just above threshold receive alerts; just below do not. Compare outcomes around discontinuity to estimate alert effect.
    \end{itemize}

\textbf{Counterfactual Impact Evaluation}: Once deployed operationally, track:
    \begin{itemize}
    \item Which districts received cascade alerts (treatment group)?
    \item Which interventions were deployed (food aid, cash transfers, livelihood programs)?
    \item IPC outcomes: Did alerted districts have better outcomes than predicted (intervention mitigated crisis)?
    \item Cost-effectiveness: What was cost per IPC phase reduction (\$/person moved from IPC 3 to IPC 2)?
    \end{itemize}

This closes the loop: prediction $\times$ alert $\times$ intervention $\times$ outcome $\times$ evaluation $\times$ model improvement.

\subsection{Multilingual News Processing: Addressing Language Bias}

GDELT's English-only coverage excludes majority of African media. Multilingual expansion needed:

\textbf{Target Languages} (by speaker population in food-insecure regions):
    \begin{enumerate}
    \item \textbf{French}: DRC, Niger, Mali, Burkina Faso, Madagascar, Chad (45\% of Africa's crisis-affected population).
    \item \textbf{Arabic}: Sudan, Somalia, Mauritania (20\%).
    \item \textbf{Portuguese}: Mozambique, Angola (8\%).
    \item \textbf{Amharic}: Ethiopia (7\%).
    \item \textbf{Swahili}: Kenya, Tanzania, DRC (5\%).
    \end{enumerate}

\textbf{Technical Approaches}:
    \begin{itemize}
    \item \textbf{Machine translation}: Translate French/Arabic/Portuguese news to English via Google Translate API, apply existing NLP pipeline (BERT embeddings, news categorisation). Cheap but introduces translation errors.

    \item \textbf{Multilingual embeddings}: Train multilingual BERT (mBERT) or XLM-RoBERTa on African news corpora \citep{conneau2020unsupervised}. Produces language-agnostic representations. Requires large training corpus (10M+ articles).

    \item \textbf{Native-language classifiers}: Train separate French-language, Arabic-language news classifiers using local corpora. Avoids translation errors but requires language-specific expertise.
    \end{itemize}

\textbf{Data Sources}:
    \begin{itemize}
    \item \textbf{AllAfrica.com}: Aggregates 1,000+ African news sources (many French, Portuguese, Arabic). Provides RSS feeds.
    \item \textbf{BBC Monitoring / Thomson Reuters Foundation}: Monitor African media in local languages, provide English summaries (paid subscription).
    \item \textbf{Local partnerships}: Collaborate with African media monitoring organisations (e.g., Institut Panos Afrique de l'Ouest, MISA - Media Institute of Southern Africa) for curated local news datasets.
    \end{itemize}

\textbf{Expected Impact}: Expanding to French/Arabic news could:
    \begin{itemize}
    \item Reduce coverage bias (currently favours Anglophone countries like Kenya, Nigeria, Zimbabwe).
    \item Increase key saves in Francophone Sahel (Niger, Mali, Burkina Faso currently have 0-12 key saves each; better coverage may improve rescue rates).
    \item Enable deployment in currently excluded countries (Chad, Central African Republic, Cameroon lack sufficient English-language coverage).
    \end{itemize}

\textbf{Research Questions}:
    \begin{enumerate}
    \item Does local-language news provide different signals than international English-language news? (Hypothesis: local news covers early-stage crises, international news covers escalated crises).
    \item Can multilingual models outperform English-only models even in Anglophone countries? (Hypothesis: yes, because regional French/Arabic news covers spillover effects from neighboring countries).
    \item What is optimal translation quality threshold for preserving news signals? (At what BLEU score does translation noise overwhelm crisis signal?).
    \end{enumerate}

\subsection{Explainable AI for Humanitarian Decision-Making: Enhanced Interpretability}

Current interpretability analysis (XGBoost importance, mixed-effects coefficients, SHAP values) serves researchers. Humanitarian practitioners need different explanations:

\textbf{Practitioner Needs}:
    \begin{itemize}
    \item \textbf{"Why did the model change its prediction for District X from low-risk to high-risk this month?"} $\times$ Need temporal explanation (which features changed between $t-1$ and $t$?).

    \item \textbf{"What evidence supports this crisis alert for District Y?"} $\times$ Need evidence synthesis (show specific news articles, extracted events, narrative regime shifts that drove prediction).

    \item \textbf{"How confident should we be in this forecast?"} $\times$ Need uncertainty quantification (prediction intervals, ensemble disagreement, data quality flags).

    \item \textbf{"Which interventions would most reduce predicted crisis risk?"} $\times$ Need counterfactual explanations (if we reduce conflict by 30\%, how much does predicted IPC improve?).
    \end{itemize}

\textbf{Enhanced Explainability Techniques}:
    \begin{enumerate}
    \item \textbf{Temporal SHAP}: Extend SHAP to time series, decompose prediction change $\Delta p = p_t - p_{t-1}$ into feature contributions: $\Delta p = \sum_i \text{SHAP}_i(\Delta x_i)$. Identifies which feature changes drove prediction shifts.

    \item \textbf{Evidential deep learning} \citep{sensoy2018evidential}: Neural network variant that outputs not just predictions but epistemic uncertainty (model uncertainty due to lack of training data) and aleatoric uncertainty (inherent randomness). Flags low-confidence predictions for expert review.

    \item \textbf{Influence functions} \citep{koh2017understanding}: Identify which training examples most influenced a specific prediction. For District X's crisis alert, show: "This prediction similar to Sudan 2021 Darfur crisis (conflict\_ratio spike + displacement\_z-score anomaly)."

    \item \textbf{Counterfactual explanations} \citep{wachter2017counterfactual}: Generate minimal feature changes that would flip prediction. "If conflict\_ratio decreased from 0.42 to 0.28 (achievable via peacekeeping deployment), predicted IPC would drop from 3.2 to 2.8 (below crisis threshold)."
    \end{enumerate}

\textbf{User Interface Design}:
    \begin{itemize}
    \item \textbf{Interactive dashboards}: Web interface showing district-level predictions, colour-coded by risk (red/orange/yellow). Click district $\times$ see feature contributions (bar charts), temporal trends (line graphs), similar historical cases (reference table).

    \item \textbf{Natural language explanations}: Auto-generate text summaries: "District X classified as high-risk (IPC 3.4 predicted) due to: (1) Conflict escalation: conflict\_ratio increased from 0.15 to 0.42 over past 3 months, indicating violence surge. (2) Narrative regime shift: HMM detected transition from peaceful to crisis-prone discourse. (3) Event extraction: Automated NER identified 15 displacement events and 8 food shortage mentions in past month."

    \item \textbf{Evidence provenance}: Hyperlink to source articles (GDELT event records), extracted crisis events, BERT semantic clusters. Enable practitioners to validate model inputs and understand text-based signals.
    \end{itemize}

\textbf{Participatory Model Evaluation}: Engage humanitarian practitioners in ongoing model validation:
    \begin{itemize}
    \item Monthly feedback sessions: Present recent predictions, ask field staff "Does this align with your assessment? What did we miss?"
    \item Disagreement analysis: When practitioners override model predictions, document rationale. Patterns inform model improvements.
    \item Co-design new features: Practitioners suggest additional data sources (e.g., "fuel price increases precede crises by 6 weeks in our experience"). Test hypothesis via feature engineering, validate via ablation studies.
    \end{itemize}

Goal: Transform black-box ML system into transparent decision support tool that amplifies human expertise rather than replacing it.




