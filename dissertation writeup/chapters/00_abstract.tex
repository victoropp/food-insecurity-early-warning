% Abstract (1 page maximum)

Food insecurity early warning systems increasingly rely on news media indicators to forecast humanitarian crises months in advance. However, existing approaches face a fundamental methodological challenge: spatio-temporal autoregressive (AR) baselines using only temporal autoregressive features ($L_t$: first-order lag of past IPC values at t-1) and spatial autoregressive features ($L_s$: inverse-distance weighted IPC values from neighboring districts)---with zero news features or external covariates---achieve AUC-ROC 0.907 at 8-month forecast horizons, approaching published news-based models that use millions of articles (93.8\% of Balashankar et al.'s PR-AUC). This \textit{autocorrelation trap} raises critical questions about when and where news signals provide genuine early-warning information beyond structural persistence in temporally and spatially autocorrelated crisis data.

This dissertation develops a two-stage residual modelling framework that addresses this challenge through: (1) rigorous spatio-temporal AR baselines using L2-regularized logistic regression with inverse-distance spatial weighting (300km radius) and stratified spatial cross-validation to identify structurally persistent crises, and (2) selective deployment of news-based models exclusively on the critical 26.8\% of crises that AR baselines miss (1,427 of 5,322 crises), where shock-driven dynamics break temporal patterns and news features drive 74.7\% of marginal predictions (SHAP analysis). This cascade approach strategically allocates computational resources: lightweight AR models for persistence-dominated cases, sophisticated news-based models for the hardest-to-predict shock-driven crises where early warning saves lives.

The empirical analysis uses 55,129 district-level IPC assessments across 24 African countries (2021-2024) and constructs interpretable dynamic features from 7.6 million GDELT news articles through four analytical stages. First, we categorise articles into nine thematic domains (conflict, displacement, economic, weather, food security, health, humanitarian, governance, other) using keyword dictionaries and calculate ratio features (compositional emphasis) and 12-month sliding-window z-score features (temporal anomalies). Second, we apply Hidden Markov Models (HMM) with 2 latent states (binary regime: Pre-Crisis vs Crisis-Prone) to detect narrative regime transitions (peaceful $\rightarrow$ violent shifts) invisible to cross-sectional aggregations, achieving 89.5\% convergence across 48-month district-level time series. Third, we employ Dynamic Mode Decomposition (DMD) to extract temporal evolution patterns, isolating crisis-relevant modes (positive growth rates indicating escalation) and filtering background modes (near-zero eigenvalues representing steady states), with 83.1\% convergence enabling mechanistic understanding of how crises unfold temporally. Fourth, we integrate these features via mixed-effects logistic regression with geographic random effects (country-level intercepts and slopes) to quantify heterogeneity in baseline risk and feature sensitivity across contexts.

We address five research questions through comprehensive ablation studies (8 feature combinations testing ratio-only, z-score-only, combined features, and HMM/DMD variants; 3,888 hyperparameter configurations via grid search; 5-fold stratified spatial cross-validation ensuring geographic separation between training/test sets) and triangulated interpretability analysis (XGBoost tree-based feature importance, mixed-effects fixed/random coefficients, SHAP game-theoretic attributions). Key findings demonstrate: \textbf{(RQ1 - The Autocorrelation Trap)} AR baselines achieve 93.8\% of Balashankar et al. (2023)'s published news-based model performance using zero text features (AR PR-AUC: 0.765 vs Balashankar PR-AUC: 0.816; AR AUC-ROC: 0.907), revealing that most published results (AUC 0.75-0.85) lacking AR comparisons may primarily reflect temporal and spatial autocorrelation rather than genuine text feature value; \textbf{(RQ2 - When News Matters)} thematic rankings exhibit measurement paradox: ratio/mixed-effects models (sustained shifts) rank weather (+26.71 coefficient) and food security (+20.33) highest, while SHAP z-score analysis (rapid anomalies) reverses rankings with conflict \#1 (0.911) and weather \#7 (0.769), demonstrating split frequency $\neq$ predictive power. Location features dominate tree splits (29.3\% cumulative) but contribute only 2.6\% marginal attribution (15.5$\times$ overstatement), serving as stratification infrastructure while z-score news features drive 74.7\% of actual predictions. This reveals complementary mechanisms: ratios capture sustained compositional changes for 8-month forecasts, z-scores detect rapid temporal anomalies for shock-driven crises. Country-specific theme elevations reveal diagnostic signals invisible to universal baselines: Zimbabwe weather coverage elevated +2.1pp above global average (drought cycles compound economic collapse), Sudan conflict +3.3pp (civil war escalation AR cannot anticipate), DRC displacement +2.2pp (M23 resurgence), Somalia health +5.8pp (disease burden compounds food insecurity)---demonstrating news features capture context-specific shock dynamics that break structural persistence patterns; \textbf{(RQ3 - Hidden Variables)} HMM provides interpretability value (hmm\_ratio\_transition\_risk ranks \#5 in feature importance at 3.2\%, capturing qualitative regime transitions invisible to compositional features) while DMD achieves largest mixed-effects coefficient (+352.38) for rare but extreme complex emergencies where multiple crisis drivers converge simultaneously (<3\% of observations); \textbf{(RQ4 - Two-Stage Framework)} the cascade rescues 249 crises (17.4\% of 1,427 AR failures) at cost of precision decline (0.732 $\rightarrow$ 0.585, -14.7pp) but favourable humanitarian cost-benefit under asymmetric weighting (10:1 FN:FP yields 6.2\% total cost reduction, prioritising recall over precision); \textbf{(RQ5 - Geographic Heterogeneity)} news features exhibit strong context-specificity, with 70.7\% of key saves concentrated in three conflict-affected countries (Zimbabwe: 77 saves, Sudan: 59, DRC: 40), country-level AUC ranging 10-fold (0.068 Niger to 0.682 Sudan), and mixed-effects random effects spanning 8.26 points (Somalia +3.70 to Madagascar -4.56), demonstrating that universal models fail and selective deployment is necessary. Within-country heterogeneity analysis reveals the same countries show both rescues and failures at district level (Zimbabwe: 77 saves but 647 still-missed, Sudan: 59 saves but 420 still-missed)$\times$news-based early warning succeeds in well-covered districts (capitals, conflict zones) but fails in news desert districts (remote pastoral areas, peripheral regions) within the same country, with rescued cases having 53\% more news coverage (121 vs 79 articles/month median).

The two-stage framework achieves recall improvement from 0.732 to 0.779 (+4.7pp, +6.4\% relative) by rescuing shock-driven crises (conflict escalation in Sudan/DRC, economic collapse in Zimbabwe, complex emergencies with simultaneous displacement/disease) where AR persistence assumptions fail. The 249 key saves represent \textit{the hardest-to-predict crises}---those invisible to spatio-temporal baselines but critical for humanitarian response. Eight months advance warning enables preemptive food assistance, livelihood support, conflict-sensitive programming, and emergency funding mobilisation before populations exhaust coping strategies.

\textbf{Critically}, cascade failure analysis reveals a fundamental constraint: the 1,178 crises still missed after cascade intervention (82.6\% of AR failures) exhibit systematic news coverage deficiency---median 74 articles/month compared to 121 for rescued cases (64\% less coverage). This \textit{news deserts hypothesis} demonstrates that news-based early warning fundamentally cannot rescue crises in remote pastoral areas (Kenya Northern, Zimbabwe rural districts, Niger) lacking sufficient media coverage. The 249 key saves concentrate in news-dense conflict zones (70.7\% in Sudan/Zimbabwe/DRC), revealing that successful cascade deployment requires rich news signal infrastructure. Remote areas with sparse coverage remain fundamentally unpredictable without expanding NLP data sources beyond traditional news media: social media monitoring (Twitter/X, Facebook, WhatsApp group analysis), community radio transcripts (local-language broadcasts), humanitarian situation reports (OCHA, UNHCR, WFP assessments), and multilingual text mining from non-English sources (Swahili, Hausa, Amharic, French, Arabic). Future NLP systems must incorporate diverse text corpora with targeted collection strategies for underreported regions.

Our methodological contributions extend beyond performance metrics to establish rigorous standards for crisis prediction research: (1) \textbf{Mandatory AR baselines} with inverse-distance spatial weighting and proper spatial cross-validation, requiring all future work claiming predictive value from external covariates to report \textit{marginal} contributions beyond autocorrelation; (2) \textbf{Prediction-interpretability trade-offs}, demonstrating that simple models (ratio+location, 12 features, AUC 0.727) maximise discrimination for difficult cases while advanced models (35 features including HMM/DMD, AUC 0.697) maximise mechanistic understanding through latent dynamics and temporal patterns; (3) \textbf{SHAP analysis exposes tree-based importance artifacts}, revealing location features account for 40.4\% of tree splits but only 2.6\% of marginal prediction attribution (15.5$\times$ overstatement), while z-score features drive 74.7\% of prediction variance despite lower tree rankings$\times$demonstrating that feature "importance" depends critically on measurement method (split frequency vs marginal impact); (4) \textbf{Evidence-based NLP deployment strategies}, providing operational guidance for selective cascade deployment in conflict zones with high news coverage (Sudan/Zimbabwe/DRC), AR-only deployment in climate contexts where spatial autoregressive features capture regional patterns (Kenya/Ethiopia pastoral zones), and recommendations for expanding text corpora in low-coverage regions (Niger/Madagascar) through social media mining, humanitarian reports, local-language sources, and community radio transcripts.

This work challenges existing literature's claims about news value for early warning, demonstrates when and where dynamic features justify computational complexity, and provides an operational framework for selective deployment that maximises humanitarian impact while respecting geographic heterogeneity. The finding that simple AR baselines achieve 90\%+ of predictive signal fundamentally reshapes understanding of what news-based forecasting can contribute, shifting focus from universal deployment to strategic targeting of shock-driven crises in news-dense contexts. By quantifying the autocorrelation trap and establishing when news features provide genuine marginal value, this dissertation sets higher methodological standards for the crisis prediction field.

\textbf{Keywords:} food insecurity, early warning systems, autocorrelation trap, two-stage residual modelling, cascade ensemble, spatio-temporal autoregression, mixed-effects regression, interpretable machine learning, GDELT, Hidden Markov Models, Dynamic Mode Decomposition, geographic heterogeneity, selective deployment, humanitarian forecasting



